%%%%%%%%%%%%%%%

%There are some problem in fonts, cant solve it yet
%update#1: maybe need CJK font download

%%%%%%%%%%%%%%%

%!TEXprogram=xelatex
%!Mode::"TeX:UTF-8"

\documentclass[12pt,a4paper]{article}
\usepackage[top=25.4mm,bottom=25.4mm]{geometry}%设置页边距
\usepackage[no-math]{fontspec}%提供字体选择命令
\usepackage{xunicode}%提供Unicode字符宏
\usepackage{xltxtra}%提供了针对XeTeX的改进并且加入了XeTeX的LOGO
\usepackage[slantfont,boldfont]{xeCJK}%使用xeCJK宏包
\usepackage{amsmath}%数学宏包
\usepackage{mathspec}%数学字体宏包
\usepackage{graphics} %图形宏包
\usepackage{titleps} %设置页眉,页脚
%\usepackage{xecolour}%找不到宏包

\usepackage{fontspec}%仿照例3
%-----------------------设置中文字体--------
%\setCJKmainfont{Adobe Song Std}%设置正文为宋体
%\setCJKmonofont{Adobe Heiti Std}%设置等距字体
%\setCJKsansfont{Adobe Kaiti Std}%设置无衬线字体

\setCJKmainfont[BoldFont=STHeiti,ItalicFont=STKaiti]{STSong}
\setCJKsansfont{STHeiti}
\setCJKmonofont{STFangsong}
 
\setCJKfamilyfont{zhsong}{STSong}
\setCJKfamilyfont{zhhei}{STHeiti}
\setCJKfamilyfont{zhfs}{STFangsong}
\setCJKfamilyfont{zhkai}{STKaiti}
 
\newcommand*{\songti}{\CJKfamily{zhsong}} % 宋体
\newcommand*{\heiti}{\CJKfamily{zhhei}} % 黑体
\newcommand*{\kaishu}{\CJKfamily{zhkai}} % 楷书
\newcommand*{\fangsong}{\CJKfamily{zhfs}} % 仿宋

%-----------------------设置英文字体------------------------
%\setmainfont[Mapping=tex-text]{TeX Gyre Pagella} %英文衬线字体
%\setsansfont[Mapping=tex-text]{Trebuchet MS} %英文无衬线字体
%\setmonofont[Mapping=tex-text]{Courier New} %英文等宽字体
%-----------------------设置数学字体--------------------------
%\setmathsfont(Digits,Latin,Greek)[Numbers={Lining,Proportional}]{MinionPro}
%%-------------------------------------------------------------
\punctstyle{kaiming}%开明式标点格式
\usepackage{indentfirst}%首段缩进
\linespread{1.5}%1.5倍行距

\title{How to type math in LaTex}
\author{Shirley}
\date{\today}
 
\begin{document}
%\maketitle

\newpagestyle{yang}{
\sethead{Sichuan University}{}{四川大学}
\setfoot{数学系}{}{第~~\thepage~~页}\headrule\footrule}\pagestyle{yang}

\begin{center}
数学公式简单示例
\end{center}

1.基本公式:
\[f(x)=2\sigma+3\]
2.积分公式:
\[ \int_a^b f(x)\,dx.\]
3.上下标:
$$\sum_{i=1}^n a_i=0$$%下标签
$$f(x)=x^{x^x}$$%上标签
4.添加公式标号:
\begin{equation}
\sigma_z=\sqrt{\Sigma(\Delta_z-\langle\Delta_z\rangle)^2/(N-1)}
\end{equation}
5.取消公式标号:
\begin{equation*}
\sigma_z=\sqrt{\Sigma(\Delta_z-\langle\Delta_z\rangle)^2/(N-1)}
\end{equation*}
6.在公式中插入文本:
$$\mbox{对任意的$x>0$}, \mbox{有 }f(x)>0. $$
7.分式及开方:
$$y=\frac{m}{n} $$ %分子,分母
$$y=\sqrt{\sigma}$$
$$y=\sqrt[n]{\lambda}$$
8.省略号:
$$f(x_1,x_x,\ldots,x_n) = x_1^2 + x_2^2 + \cdots + x_n^2 $$
%\ldots 表示跟文本底线对齐的省略号;\cdots 表示跟文本中线对齐的省略号
9、括号和分隔符:
% () 和 [ ] 和 | 对应于自己;
% {} 对应于 \{ \};
% || 对应于 \|。
% 当要显示大号的括号或分隔符时,要对应用 \left 和 \right
\[f(x,y,z) = 3y^2 z \left( 3 + \frac{7x+5}{1 + y^2} \right).\]
%\left. 和 \right. 只用与匹配,本身是不显示的:
 $$\left. \frac{du}{dx} \right|_{x=0}=2x$$
 10.多行对齐公式:
 \begin{eqnarray*}
\cos 2\theta & = & \cos^2 \theta - \sin^2 \theta \\
& = & 2 \cos^2 \theta - 1.
\end{eqnarray*}
 \begin{eqnarray}
\cos 2\theta & = & \cos^2 \theta - \sin^2 \theta \\
& = & 2 \cos^2 \theta - 1.
\end{eqnarray}
% 其中&是对其点,表示在此对齐。
11.矩阵:\\
The \emph{characteristic polynomial} $\chi(\lambda)$ of the
$3 \times 3$~matrix
\[ \left( \begin{array}{ccc}
a & b & c \\
d & e & f \\
g & h & i \end{array} \right)\]
is given by the formula
\[ \chi(\lambda) = \left| \begin{array}{ccc}
\lambda - a & -b & -c \\
-d & \lambda - e & -f \\
-g & -h & \lambda - i \end{array} \right|.\]
% c表示向中对齐,l表示向左对齐,r表示向右对齐。
12、导数、极限、求和、积分(Derivatives, Limits, Sums and Integrals):
$$\frac{du}{dt} and \frac{d^2 u}{dx^2}$$
\[ \frac{\partial u}{\partial t}
= h^2 \left( \frac{\partial^2 u}{\partial x^2}
+ \frac{\partial^2 u}{\partial y^2}
+ \frac{\partial^2 u}{\partial z^2}\right)\]
$$\lim_{x \to +\infty}, \inf_{x > s} and \sup_K$$
\[ \lim_{x \to 0} \frac{3x^2 +7x^3}{x^2 +5x^4} = 3.\]
\[ \sum_{k=1}^n k^2 = \frac{1}{2} n (n+1).\]
\[ \int_a^b f(x)\,dx.\]
\[ \int_0^{+\infty} x^n e^{-x} \,dx = n!.\]
\[ \int \cos \theta \,d\theta = \sin \theta.\]
\[ \int_{x^2 + y^2 \leq R^2} f(x,y)\,dx\,dy
= \int_{\theta=0}^{2\pi} \int_{r=0}^R
f(r\cos\theta,r\sin\theta) r\,dr\,d\theta.\]
\[ \int_0^R \frac{2x\,dx}{1+x^2} = \log(1+R^2).\]

One would typeset this in LaTeX by typing In non-relativistic wave mechanics, the wave function
$\psi(\mathbf{r},t)$ of a particle satisfies the
\emph{Schr\"{o}dinger Wave Equation}
\[ i\hbar\frac{\partial \psi}{\partial t}
= \frac{-\hbar^2}{2m} \left(
\frac{\partial^2}{\partial x^2}
+ \frac{\partial^2}{\partial y^2}
+ \frac{\partial^2}{\partial z^2}
\right) \psi + V \psi.\]

It is customary to normalize the wave equation by
demanding that
\[ \int \!\!\! \int \!\!\! \int_{\textbf{R}^3}
\left| \psi(\mathbf{r},0) \right|^2\,dx\,dy\,dz = 1.\]

A simple calculation using the Schr\"{o}dinger wave
equation shows that
\[ \frac{d}{dt} \int \!\!\! \int \!\!\! \int_{\textbf{R}^3}
\left| \psi(\mathbf{r},t) \right|^2\,dx\,dy\,dz = 0,\]
and hence
\[ \int \!\!\! \int \!\!\! \int_{\textbf{R}^3}
\left| \psi(\mathbf{r},t) \right|^2\,dx\,dy\,dz = 1\]
for all times~$t$. If we normalize the wave function in this
way then, for any (measurable) subset~$V$ of $\textbf{R}^3$
and time~$t$,
\[ \int \!\!\! \int \!\!\! \int_V
\left| \psi(\mathbf{r},t) \right|^2\,dx\,dy\,dz\]
represents the probability that the particle is to be found
within the region~$V$ at time~$t$.

\end{document}

%%%%%%%%%%%%%%%%%%%%%%%%%%%%%%%%%%%%%%%%%%%%%

%\documentclass[12pt,a4paper]{article}  
%\usepackage{fontspec, xunicode, xltxtra}  
%\setmainfont{Heiti SC}  
%\begin{document}  
%Hello,World!  
%这是一个测试的中文文档!  
%\end{document} 

%%%%%%%%%%%%%%%%%%%%%%%%%%%%%%%%%%%%%%%%%%%%%

%测试成功,说明上面还是字体原因
%\documentclass{article}
%\usepackage{fontspec}
%\setromanfont{LiHei Pro} % 儷黑Pro
%\setmonofont{Courier New} % 等寬字型
%\XeTeXlinebreaklocale "zh"
%\XeTeXlinebreakskip = 0pt plus 1pt
%\begin{document}
%在Mac下的XeTeX裡寫中文~
%\end{document}

%%%%%%%%%%%%%%%%%%%%%%%%%%%%%%%%%%%%%%%%%%%%%


%\documentclass[19pt]{article}
%\usepackage{fontspec}
%\setmainfont{Songti SC}
%\title{无题}
%\author{李商隐}
%\date{}
%
%\begin{document}
%\maketitle
%
%\begin{center}
%相见时难别亦难\
%东风无力百花残\
%\end{center}
%
%\end{document}





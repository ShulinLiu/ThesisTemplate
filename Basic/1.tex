\documentclass[12pt,openany,CJK,oneside]{cctbook}
\usepackage{mathrsfs}
\usepackage{amsmath}
\usepackage{amsthm}
\usepackage{cases}
\usepackage{amssymb}
\usepackage{latexsym}
\usepackage{graphicx}

\usepackage[footnotesize]{subfigure}
\usepackage[footnotesize]{caption2}
\usepackage{color}
\usepackage{epsf}
\usepackage{fancyhdr}

\usepackage{remreset}

\allowdisplaybreaks
%%%%%%%%%%%%%%%%%%%%%%%%%%%%%
\renewcommand{\theequation}{\arabic{chapter}.\arabic{equation}}
\renewcommand{\thefigure}{\arabic{figure}}
\renewcommand{\thetable}{\arabic{table}}

\makeatletter
\@removefromreset{table}{chapter}
\@removefromreset{figure}{chapter}
\makeatother
\renewcommand{\thetable}{\arabic{table}}
\renewcommand{\thefigure}{\arabic{figure}}
\newtheorem{defn}  {{\heiti ¶šÒå}}[section]
\newtheorem{lem}  {{\heiti ÒýÀí}}[chapter]
\newtheorem{thm} {{\heiti ¶šÀí}}[chapter]
\newtheorem{exam}{{\heiti Àý  }}[section]
\newtheorem{rem}  {{\heiti ×¢  }}[section]
\newtheorem{cor}{{\heiti ÍÆÂÛ}}[section]
\newtheorem{prop}{{\heiti ÃüÌâ}}[section]

\newcommand{\norm}[1]{\left\Vert#1\right\Vert}
\newcommand{\abs}[1]{\left\vert#1\right\vert}
\newcommand{\set}[1]{\left\{#1\right\}}
\newcommand{\Real}{\mathbb R}
\newcommand{\eps}{\varepsilon}
\newcommand{\To}{\longrightarrow}
\newcommand{\BX}{\mathbf{B}(X)}
\newcommand{\A}{\mathcal{A}}
\newcommand{\rbk}[1]{\left(#1\right)}
\newcommand{\bk}[1]{\left[#1\right]}
\newcommand{\hp}[1]{\hphantom{#1}}
\newcommand{\inn}[2]{#1,\cdots,#2}

\newcommand{\Sum}[4]{\sum
      _{
      \begin{subarray}{l}
        #1=#2 \\
        #1 \neq #4
      \end{subarray}
      }^#3
      }

\newcommand{\Max}[3]{\max_{#2 \leq \,#1 \leq \,#3}}
\newcommand{\ol}[1]{\overline{#1}}
\newcommand{\dfi}{\stackrel{\textstyle\vartriangle}{=}}
\newcommand{\sech}{\ensuremath{\mathrm{sech\ }}}

\makeatletter
\renewenvironment{proof}[1][Ö€Ã÷]{\par
    \pushQED{\qed}%
    \normalfont \topsep6\p@\@plus6\p@\relax
    \trivlist
    \item[\hskip\labelsep
        \itshape\bfseries
      #1\@addpunct{}]\ignorespaces
}{%
    \popQED\endtrivlist\@endpefalse
} \makeatother

\makeatletter
\def\@captype{figure}
\makeatother


\DeclareMathOperator{\diag}{diag} \DeclareMathOperator{\dist}{dist}
\DeclareMathOperator{\col}{col} \DeclareMathOperator{\sgn}{sgn}

\topmargin 2 cm \setlength{\parindent}  {2\ccwd}
\setlength{\parskip}    {3pt plus1pt minus2pt}
\setlength{\baselineskip}    {20pt plus2pt minus1pt}
\setlength{\textheight} {21.5true cm}                 %21.5cm
\setlength{\textwidth}  {14.5true cm}               %14.5cm
\pagestyle{empty}
                                  %È¡ÏûÒ³Âë

\begin{document}
\begin{titlepage}
\vspace{5cm}
\begin{center}{\Huge \biaosong ËÄ\ Žš\ Žó\ ѧ\\
\vspace{1.2cm} ˶\ Ê¿\ ѧ\ λ\ ÂÛ\ ÎÄ}\end{center}


\vspace{2.3cm}
\begin{quote}
{\Large {\fangsong \hspace{0.25cm}Ìâ\ Ä¿}\hspace{0.15cm}
\underline{{\kaishu \hspace{0.85cm} ÎÞ×èÄᵯÌøÇòÄ£Ð͵ÄÍêÕûÕð²üÏÖÏó \hspace{0.85cm}}} }
\end{quote}

\vspace{0.8cm}

\begin{quote}
{\Large {\fangsong \hspace{0.25cm}×÷\ Õß}  \underline{\hspace{0.1cm}\kaishu\ \ \ \ Óô\ \ ŒÑ\ \ \ \ \hspace{0.15cm}}
 \hspace{0.1cm}\fangsong Íê³ÉÈÕÆÚ\underline {\hspace{0.5cm}{\large 2015}\hspace{0.15cm}{\fangsong Äê}\hspace{0.15cm}{\large 3}\hspace{0.1cm}{\fangsong ÔÂ}\hspace{0.1cm}{\large 12}\hspace{0.1cm}{\fangsong ÈÕ}\hspace{0.5cm}}}\\
\vspace{2.5cm}
\begin{quote}
       {\Large {\fangsong \hspace{0.45cm} Åà\hspace{0.38cm}Ñø\hspace{0.38cm}µ¥\hspace{0.38cm}λ}
               \underline{\hspace{1.9cm} \kaishu ËÄ\ \ Žš\ \ Žó\ \ ѧ \hspace{1.9cm}}\vspace{0.9cm}
       \\
               \vspace{0.9cm}    {\fangsong \hspace{0.45cm}  Öž\hspace{0.38cm} µŒ\hspace{0.38cm}œÌ\hspace{0.38cm}ÊŠ}
               \underline{\hspace{1.8cm}  \kaishu ¶Å\ Õý\ ¶« \ \ œÌÊÚ \hspace{1.8cm}}
       \\
               \vspace{0.9cm}    {\fangsong  \hspace{0.45cm} ך\hspace{2.3cm}ҵ}
               \underline{\hspace{1.9cm}  \kaishu  »ù\ \ Ž¡\ \ Êý\ \ ѧ \hspace{1.9cm}}
       \\
               \vspace{0.9cm}    {\fangsong \hspace{0.45cm} ÑÐ\hspace{0.38cm}Ÿ¿\hspace{0.38cm}·œ\hspace{0.38cm}Ïò}
               \underline{\hspace{0.4cm} \kaishu \ \ \ ΢·Ö·œ³ÌÓ붯ÁŠÏµÍ³ \hspace{1.1cm}}
       \\
               \vspace{0.9cm}    {\fangsong \hspace{0.45cm} ÊÚÓÚѧλÈÕÆÚ}
               \underline{\hspace{1.8cm}  {\fangsong Äê}    \hspace{1.21cm} {\fangsong ÔÂ}  \hspace{1.21cm} {\fangsong ÈÕ}\hspace{0.4cm}}}
\end{quote}
 \end{quote}
\end{titlepage}
%%%%%%%%%%%%%%%%%%%%%%%%%%%%%
%\cleardoublepage
\setlength{\evensidemargin}{0.88cm}
\setlength{\oddsidemargin}{0.88cm}
\parskip 10pt
\baselineskip 20pt
\vspace{1.2cm}

\chapter*{\heiti ÖÂ\ \ \ л }
%\addcontentsline{toc}{chapter}{\numberline{}\mbox{\heitiÖÂл}}

»ùÓÚWigner-Ville·Ö²ŒµÄ˲ʱƵÂʹÀŒÆ£º
$f\left( t \right)=\frac{\int_{R}{uW\left( t,u \right)du}}{\int_{R}{W\left( t,u \right)du}}$.
ÀàËƵأ¬¿ÉÒԵõœ»ùÓÚÏßÐÔÕýÔòÓòµÄWigner-Ville·Ö²ŒµÄ˲ʱƵÂʹÀŒÆ¡£
¶šÀí£º ŒÙÉèÐźÅÊÇ\[x\left( t \right)=a\left( t \right){{e}^{j\phi \left( t \right)}}\], ²¢ÇÒ\[\phi \left( t \right)\]isÊÇÐźŵÄÏà룬ÄÇÃŽÐźÅ\[x\left( t \right)\]»ùÓÚÏßÐÔÕýÔòÓòµÄWigner-Ville·Ö²ŒµÄ˲ʱƵÂʹÀŒÆÊÇ£º
\begin{align}{{f}_{IF}}\left( t \right)=\frac{1}{2\pi }\frac{d\phi \left( t \right)}{dt}=\frac{1}{2\pi }\frac{\int_{R}{\frac{u}{b}W_{A}^{x}\left( t,u \right){{e}^{-j\frac{d{{u}^{2}}}{2b}}}du}}{\int_{R}{W_{A}^{x}\left( t,u \right){{e}^{-j\frac{d{{u}^{2}}}{2b}}}du}}
\end{align}            (6)
Ö€Ã÷.
¶ÔÐźÅ\[x\left( t \right)=a\left( t \right){{e}^{j\phi \left( t \right)}}\]Áœ±ßÇ󵌵õœ£º
\[{x}'\left( t \right)=ja\left( t \right){\phi }'\left( t \right){{e}^{j\phi \left( t \right)}}+{a}'\left( t \right){{e}^{j\phi \left( t \right)}}\].               (7)
ÔÚµÈÊœ(7)Áœ±ßͬʱ³ËÒÔÐźÅ\[x\left( t \right)\]µÄ¹²éîÐźÅ\[{{x}^{*}}\left( t \right)\]
\[{x}'\left( t \right){{x}^{*}}\left( t \right)=j{{a}^{2}}\left( t \right){\phi }'\left( t \right)+{a}'\left( t \right)a\left( t \right)\],               (8)
²¢ÇÒŽÓµÈÊœ(8)ÖпÉÒԵõœ
\[{\phi }'\left( t \right)=\frac{\operatorname{Im}\left[ {x}'\left( t \right){{x}^{*}}\left( t \right) \right]}{{{a}^{2}}\left( t \right)}=\frac{\operatorname{Im}\left[ {x}'\left( t \right){{x}^{*}}\left( t \right) \right]}{\left| {{x}^{2}}\left( t \right) \right|}\].             (9)
    žùŸÝµÈÊœ£š1£©ºÍ (9)
\[{{f}_{IF}}\left( t \right)=\frac{{\phi }'\left( t \right)}{2\pi }=\frac{\operatorname{Im}\left[ {x}'\left( t \right){{x}^{*}}\left( t \right) \right]}{2\pi \left| {{x}^{2}}\left( t \right) \right|}\].              (10)
»ùÓÚµÈÊœ(5)¿ÉµÃ
\[{{R}_{x}}\left( t,\tau  \right)=x\left( t+\frac{\tau }{2} \right){{x}^{*}}\left( t-\frac{\tau }{2} \right)=\int_{R}{W_{A}^{x}\left( t,u \right){{K}_{{{A}^{-1}}}}\left( u,\tau  \right)du}\].    (11)
    ÔÚµÈÊœ(11)ÖУ¬Áî\[\tau =\text{0}\]£¬¿ÉµÃ
\[x\left( t \right){{x}^{*}}\left( t \right)=\left| {{x}^{2}}\left( t \right) \right|=\frac{1}{\sqrt{-j2\pi b}}\int_{R}{W_{A}^{x}\left( t,u \right){{e}^{-j\frac{d{{u}^{2}}}{2b}}}du}\].       (12)
    ÔÚµÈÊœ(11) ÖжÔ\[\tau \]Ç󵌣¬ÇÒÁî\[\tau =\text{0}\]µÃµœ

\thispagestyle{empty}
{\Large \kaishu

±ŸÎÄÔÚÑ¡ÌâºÍÑП¿¹ý³ÌÖжŒµÃµœÎҵĵŒÊŠ¶ÅÕý¶«œÌÊÚµÄÏ€ÐÄÖžµŒ. ¶ÅÀÏÊŠ¶àŽÎѯÎÊÑП¿œø³Ì, ²¢ÖžµãÃÔœò, ¿ªÍØˌ·. ÎÒÖÔÐÄžÐл¶ÅÕý¶«ÀÏÊŠµÄÄÍÐĜ̻å, ²»ŽÇÐÁ¿àµÄÐÁÇÚÔÔÅà. ¶ÅÀÏÊŠÑÏœ÷ÇóʵµÄÖÎѧ̬¶È, ¶ÔѧÊõµÄÃôÈñ¶Ž²ìÁŠºÍÕýÖ±Žï¹ÛµÄŽŠÊÂ̬¶È, ²»œöÊÚÎÒÒÔÎÄ, ¶øÇÒœÌÎÒ×öÈË, ËäÀúÈýÔØ, ÈŽžøÎÒÒÔÖÕÉúÊÜÒæÖ®µÀ. œèŽËÂÛÎÄÍê³ÉÖ®ŒÊ, œ÷ÏòµŒÊŠ¶ÅÕý¶«œÌÊÚÖÂÒÔÖÔÐĵÄлÒâºÍ³çžßµÄŸŽÒâ.

žÐлÕÅΰÄêœÌÊÚ¡¢Ðì±ùœÌÊÚ¡¢³ÂÐËÎäž±œÌÊÚ. ÕÅÀÏÊŠ¡¢ÐìÀÏÊŠºÍ³ÂÀÏÊŠÔÚ¶¯ÁŠÏµÍ³ÌÖÂÛ¿ÎÉÏ·çÈ€ÓÄĬ¡¢ÓÉdzÈëÉîµÄœÌѧ·çžñ, ¶ÔŽýѧÊõÈÏÕæÑÏœ÷µÄ×÷·ç, ¶ÔŽýÉú»îÈÈÇéµÄ̬¶È, ÎÒ¶Œ»ñÒæÁŒ¶à.

žÐлÁõŸŽ»ª¡¢žß¿¡Ã÷ÊŠÐÖ¡¢ÒÔŒ°ÎÒµÄͬѧ³Â¿­¡¢ËŸÎÄ, »¹ÓÐÊŠµÜ³ÂË«ÔÚѧϰºÍÉú»îÖжÔÎÒµÄÈÈÐĵÄÕչ˺ÍÎÞËœµÄ°ïÖú, ÈÃÎÒÔÚÒ»žö³äÂúÎÂÜ°µÄ»·Ÿ³ÖжȹýÁËÈýÄêµÄÑП¿ÉúÉú»î.

×îºó, žÐлÎҵČÒÈË, žÐлžžÄžžøÓèÎÒÎÞËœµÄ°®ºÍÈËÉúµÄÖžÒý, žÐлÎҵĞçžçºÍÉ©×ÓÔÚÎÒÇóѧÆÚŒä¶ÔžžÄžµÄÕÕ¹Ë, ÊÇÄãÃǵÄÖ§³ÖʹÎÒ˳ÀûÍê³Éѧҵ.

}


\def\beq{\begin{equation}}
\def\eeq{\end{equation}}
\makeatletter \@addtoreset{equation}{section} \makeatother

\cleardoublepage
\parskip 10pt
\baselineskip 20pt \frontmatter \thispagestyle{empty}
\vspace*{0.3cm}
\begin{center}{\Large \heiti ÎÞ×èÄᵯÌøÇòÄ£Ð͵ÄÍêÕûÕð²üÏÖÏó}\\
\vspace{0.2cm}
»ùŽ¡Êýѧךҵ\\
\vspace{0.2cm} {\heiti ÑП¿Éú\ } {\kaishu ÓôŒÑ} \qquad {\heiti
ÖžµŒœÌÊŠ\ } {\kaishu ¶Å Õý ¶«}\end{center}

\vspace{0.7cm}
ÔںܶàʵŒÊÎÊÌâÖÐÍùÍùÓöµœÉ挰Åöײ¡¢ÄŠ²ÁÒÔŒ°Çл»µÄϵͳ, ÕâÀàϵͳһ°ãÊǷֶι⻬ϵͳ, ÒòŽË, ÔÚ¹€³ÌÁŠÑ§¡¢¿ØÖÆÀíÂÛµÈѧ¿ÆµÄÍƶ¯ÏÂ, ¶Ô·Ö¶Î¹â»¬ÏµÍ³µÄÑП¿ÒѳÉΪœüÄêÀŽµÄÒ»žöÈȵã. ·Ö¶Î¹â»¬ÏµÍ³ŸßÓÐÊ®·ÖžŽÔӵĶ¯ÁŠÑ§ÐÐΪ, ²»œö°üº¬Á˹⻬ϵͳËùŸßÓеğ­µä·Ö²íÏÖÏó, »¹ÓкܶàÓɷǹ⻬ÐÔËùÓÕµŒµÄÌØÊâ·Ö²íÏÖÏó, Èç²ÁÅö¡¢±ßœçÅöײ¡¢»¬¶¯Ð§ÓŠºÍÕð²üµÈ. ÆäÖÐÕð²üÏÖÏóÊÇÒ»Àà·Ç³£ÖØÒªµÄ·Ç¹â»¬·Ö²íÏÖÏó.

µ¯ÌøÇòÄ£ÐÍÊÇÒ»žöµäÐ͵ķֶι⻬ϵͳ, ×îÔçÓÉ·ÑÃ×Ìá³ö, ÔÚÎïÀíѧÉÏÓзdz£ÖØÒªµÄÓŠÓÃ. žÃÄ£ÐÍÐÎÊœËäÈ»Œòµ¥, µ«¶¯ÁŠÑ§ÐÐΪȎʮ·Ö·áž». Ëü³ýÁËŸßÓа°œáµã·Ö²í¡¢±¶ÖÜÆÚ·Ö²íµÈŸ­µä·Ö²íÏÖÏóÍâ, »¹ÓкܶàÓɷǹ⻬ÐÔÓÕµŒµÄÌØÊâ·Ö²íÏÖÏó, Èç²ÁÅö¡¢±ßœçÅöײºÍÕð²üµÈ.

±ŸÎÄœ«ÌÖÂÛÎÞ×èÄᵯÌøÇòÄ£Ð͵ÄÍêÕûÕð²üÏÖÏó, ŒŽÔÚÓÐÏÞµÄʱŒäÄÚ, µ¯ÌøÇòÓë×öÖÜÆÚÐÔÕñ¶¯µÄÆœ°å·¢ÉúÎÞÊýŽÎÅöײ²¢Ç÷ÓÚŸ²Ö¹µÄÏÖÏó. ÀûÓÃÖÐÖµ¶šÀí, žø³öÁËÎÞ×èÄᵯÌøÇòÄ£ÐÍÔÚ·Žµ¯ÏµÊý·Ö±ðÇ÷ÓÚ\ $0$ ºÍÇ÷ÓÚ\ $1$ ʱ·¢ÉúÍêÕûÕð²üʱ, ϵͳ²ÎÊýÓë³õʌֵËùÂú×ãµÄ³ä·ÖÌõŒþ. ²¢œøÐÐÁËÏàÓŠµÄÊýֵģÄâ, ÊýֵģÄâµÄœá¹ûºÍÎÒÃǵÄÀíÂÛÔ€²âÍêÈ«ÎǺÏ. ͬʱÔÚµ¯ÌøÇòÄ£ÐÍ·¢ÉúÖÜÆÚÐÔÕð²üµÄʱºò, µ±ÖÜÆÚÐÔÕð²üµÄÌõŒþœÓœüÁÙœç׎̬ʱ, ÀûÓÃ\ Taylor ¹«Êœžø³öÁËÕð²üʱŒäÓëϵͳ²ÎÊýŒäµÄœüËƹÀŒÆÊœ.

±ŸÎĵÚÒ»ÕÂÐ÷ÂÛœéÉÜÁ˷ֶι⻬ϵͳµÄ·Ö²íÒÔŒ°±ŸÎĵÄÖ÷ÒªÑП¿¹€×÷, µÚ¶þÕÂœéÉÜÁ˵¯ÌøÇòÄ£Ð͵ĶþάӳÉäµÄÓÉÀŽŒ°ÍêÕûÕð²ü¡¢Õð²üʱŒäµÄ¶šÒåºÍ±ŸÎĵÄÖ÷Òªœá¹û, µÚÈýÕÂÀûÓÃÖÐÖµ¶šÀí, žø³öÁ˱ŸÎÄÖ÷Òªœá¹ûµÄÖ€Ã÷, µÚËÄÕÂÓÃÊýֵģÄâµÄ·œ·šÑéÖ€Á˱ŸÎĵÄÖ÷Òªœá¹û, ×îºóµÚÎåÕÂΪœáÊøÓï.


{\bf ¹ØŒüŽÊ£º} {·Ö¶Î¹â»¬¶¯ÁŠÏµÍ³, µ¯ÌøÇò, Õð²ü, ÖÐÖµ¶šÀí.}
\newpage
\parskip 10pt
\baselineskip 20pt \thispagestyle{empty} \vspace*{0.2cm}
\begin{center}{\bf \Large Complete chattering behavior of elastic bouncing ball}

\vspace{0.2cm} {\small
{\bf Major£º} Pure Mathematics\\
\vspace{0.2cm} {\bf Graduate Student£º} YU Jia  \qquad {\bf
Supervisor£º} DU Zhengdong} \end{center} \vspace{0.4cm}

There are many real world problems involving collisions, frictions and switching components. They are often modeled by piecewise smooth dynamical systems. Consequently, motivated by applications from mechanics, electrical engineering and control theory, the study of bifurcation phenomena in piecewise smooth systems has become very popular in recent years. It is well known that, the dynamics of piecewise smooth systems are very complicated. Besides the occurrence of all kinds of traditional bifurcations, such as saddle-node bifurcation, Hopf bifurcation as well as homoclinic bifurcation, period doubling bifurcation, there are many discontinuity induced new types of complicated bifurcation phenomena, such as grazing, border-collision, sliding effects, sticking and chattering. Chattering is a very important such nonstandard phenomena.

The bouncing ball system is a typical example of piecewise smooth dynamical systems. It was first proposed by Fermi and has many applications in physics and mechanics. Although the governing equations of this system are very simple, its dynamical behaviors are very rich and complicated. It can undergo classical bifurcations, such as saddle-node bifurcation and period doubling bifurcation, as well as discontinuity induced bifurcation phenomena such as grazing, sticking and chattering. As a result, it continues to be analyzed by many papers.

In this paper we discuss the complete chattering behavior of an elastic bouncing ball without damping on a vibrating platform. Here complete chattering means that the bouncing ball performs infinity of smaller and smaller bounces with the vibrating platform in a finite time. By using the Mean Value Theorem, we obtain sufficient conditions in terms of parameters and initial values of the system under which complete chattering occurs as the restitution coefficient approaches 0 and 1 respectively. Numerical simulations that validate the theoretical results are given. Using Taylor series expansion, we also give an asymptotic estimation of the chattering time with respect to system parameters when the bouncing ball undergoes periodic chattering.

This paper is organized as follows. In Chapter 1, we briefly survey some progress of piecewise smooth dynamical systems and the study of bouncing ball and describe the main works of the paper. We introduce a two dimensional map of bouncing ball system and the notion of complete chattering and present the main results in Chapter 2. The proofs of the main results by the Mean Value Theorem are given in Chapter 3. Numerical simulations that validate the theoretical results are given in Chapter 4. Concluding remarks are given in Chapter 5.

{\bf Keywords: } {piecewise smooth dynamical system; bouncing ball; chattering; mean value theorem.}
%%%%%%%%%%%%%%%%%%%%%%%%%%%%%%%%%%%%%%%%%%%%%%%%%%%%%%%%%%%%%%%%%%%%%%%%%%%%%%%%%%%%%
\clearpage {\setlength{\parskip}{1pt} \tableofcontents
%\addcontentsline{toc}{chapter}{\numberline {}\mbox{\heiti žœÂŒ}}
\thispagestyle{empty}
\clearpage
%\makeatletter
%\let \asas \ps@plain
%\let \ps@plain \ps@empty
%\makeatother
%\pagestyle{empty}
%
%\tableofcontents
%
%\makeatletter
%\let \ps@plain \asas
%\let\asas\relax
%\makeatother
%\clearpage
%\pagestyle{fancy}
%\fancyhf{}
%\fancyhead[LE,RO]{\thepage}
%\fancyhead[LO]{\it\leftmark}
%\fancyhead[RE]{XXX}


%%%%%%%%%%%%%%%%%%%%%%%%%%%%%%%%%%%%%%%%%%%%%%%%%%%%%%%%%%%%%%%%%%%%%%%%%%%
\makeatletter
\def\@evenhead{\pushziti\vbox{\hbox to
\textwidth{\hfil{\rightmark\hfil{\rm{µÚ\,\thepage\,Ò³}}}}\protect\vspace{2
truemm}\relax \hrule depth0pt height 0.15truemm  width
\textwidth}\popziti}
\def\@oddhead{\pushziti\vbox{\hbox to \textwidth{\hfil{\rightmark\hfil{\rm{µÚ\,\thepage\,Ò³}}}}
\protect\vspace{2 truemm} \relax\hrule depth0pt height0.15truemm
width \textwidth }\popziti}
\def\@evenfoot{}
\def\@oddfoot{}
\makeatother
\mainmatter
\chapter{\heiti Ð÷ÂÛ}
\section{\kaishu ·Ö¶Î¹â»¬ÏµÍ³µÄ·Ö²íŒ°ÆäÑП¿ÏÖ׎}
·Ö¶Î¹â»¬ÏµÍ³ÔںܶàÉ挰Åöײ¡¢ÄŠ²ÁÒÔŒ°Çл»µÄ¹€³ÌʵŒÊÎÊÌâÖкܳ£Œû, ±ÈÈçŽòÓ¡Žž \;\cite{Hendricks}, žÕ¿éÔ˶¯ \;\cite{Hogan1} ºÍ²œÐлú \;\cite{Holmes} µÈµÈ. œüÄêÀŽ¹ØÓڷֶι⻬¶¯ÁŠÏµÍ³µÄÑП¿ÔœÀŽÔœÊܵœÈËÃǵÄÖØÊÓ, ÒѳÉΪ΢·Ö·œ³ÌÓ붯ÁŠÏµÍ³ÁìÓòµÄÒ»žöÑП¿Èȵã. ÑП¿±íÃ÷, ·Ö¶Î¹â»¬¶¯ÁŠÏµÍ³µÄ¶¯ÁŠÑ§ÐÐΪʮ·ÖžŽÔÓ, ÆäÖаüº¬°°œáµã·Ö²í(saddle-node bifurcation), Òô²æ·Ö²í(pitchfork bifurcation), ±¶ÖÜÆÚ·Ö²í(period doubling bifurcation), Hopf Œ°Í¬ËÞ·Ö²í(homoclinic bifurcation)\;\cite{Chow, Shaw1, Shaw2, Shaw3}, ÕâЩ·Ö²íÏÖÏóÔڹ⻬¶¯ÁŠÏµÍ³ÖÐÒ²»á³öÏÖ, µ«»¹ÓкܶàÌØÊâµÄÏÖÏóÊǷֶι⻬¶¯ÁŠÏµÍ³¶ÀÓеÄ, ±ÈÈç: ²Á±ß(grazing)¡¢ ±ßœçÅöײ(border-collision)¡¢»¬¶¯Ð§ÓŠ(sliding effect)¡¢Õð²ü(chattering) ºÍÕ³Ìù(sticking) µÈµÈ \;\cite{BuddL, Casas, DemeioL, Foale, Nordmark1}. ÓÉÓڷֶι⻬ϵͳµÄžŽÔÓÐÔ, ÔÚʵŒÊÑП¿Ê±, ÊýֵģÄâºÍ¶šÐÔ·ÖÎöÏàœáºÏµÄ·œ·šŸ­³£±»²ÉÓÃ, Brandon ºÍ\ Ueta µÈÈËÔÚÎÄÏ×\ \cite{Brandon} Öй¹ÔìÁËÒ»ÖÖÊýÖµŒÆËã·œ·šÀŽ·ÖÎö·Ö¶Î¹â»¬¶¯ÁŠÏµÍ³µÄ·Ö²íÏÖÏó.

Èçͬ¹â»¬¶¯ÁŠÏµÍ³Ò»Ñù, È·¶š·Ö¶Î¹â»¬¶¯ÁŠÏµÍ³µÄŒ«ÏÞ»·µÄÊýÁ¿ºÍλÖÃÊÇŒ«ÆäÖØÒªµÄÑП¿¿ÎÌâ. œüÄêÀŽºÜ¶àѧÕßÔÚÕâ·œÃæ×öÁËŽóÁ¿¹€×÷, È¡µÃÁ˺ܶà³É¹û. ÔÚÏÖÓеÄÎÄÏ×ÖÐ, ÓÐÈýÖÖ¹¹Ôì·Ö¶Î¹â»¬ÏµÍ³¶àžöŒ«ÏÞ»·µÄ·œ·š. µÚÒ»ÖÖ·œ·šÊÇÔÚ²»Á¬ÐøÆœÃæϵͳÖÐ, µ±·¢Éú\ Hopf ·Ö²í»òÖÐÐÄ·Ö²íʱ, ÔÚÆæµãžœœü»á²úÉúСÕñ·ùµÄŒ«ÏÞ»·. Coll, Gasull ºÍ\ Prohens \cite{Coll}, Han ºÍ Zhang \cite{Han}, Leine ºÍ Nijmeijer\ \cite{Leine}, Zou ºÍKš¹pper\ \cite{Zou} ÑП¿Á˷ֶι⻬¶¯ÁŠÏµÍ³ÔÚ²»Í¬ÇéÐÎϵÄHopf ·Ö²íÏÖÏó; Gasull ºÍTorregrosa ÌÖÂÛÁËÆœÃæ·Ö¶Î¹â»¬×ÔÖÎϵͳµÄÖÐÐÄœ¹µãµÄÅжšÎÊÌâºÍÓÉÏàÓŠµÄ\ Hopf ·Ö²í²úÉúŒ«ÏÞ»·µÄžöÊýÎÊÌâ\ \cite{Gasull}; Akhmet ºÍ \ Aru\v{g}aslan ÔÚÎÄÏ×\ \cite{Akhmet} ÖÐÑП¿ÁËÆœÃæ\ Filippov ϵͳŽøÓÐÓÐÏÞÌõÇл»ÏßʱµÄÖÐÐÄœ¹µãÅжšºÍ\ Hopf ·Ö²íÏÖÏó; Chen ºÍ\ Du ÔÚÎÄÏ×\ \cite{Chen} ÖÐÖ€Ã÷ÁËÒ»Àà¶þŽÎ²»Á¬ÐøϵͳµÄÖÐÐÄ¿ÉÒÔÖÁÉÙ·Ö²í³ö\ 9 žöŒ«ÏÞ»·; Huan ºÍ\ Yang ÔÚÎÄÏ×\ \cite{Huan} ÖÐÑП¿ÁËÒ»°ãµÄÆœÃæ·Ö¶ÎÏßÐÔϵͳµÄŒ«ÏÞ»·µÄžöÊýÎÊÌâ. µÚ¶þÖÖ¹¹Ô쌫ÏÞ»·µÄ·œ·šÊÇÓÉÖÜÆÚ»·Óò·Ö²í²úÉúŽóÕñ·ùŒ«ÏÞ»·. Du, Li ºÍ\ Zhang ÔÚÎÄÏ×\ \cite{DuLi,Dulizhang0} ÖÐÑП¿ÁËÆœÃæ\ Filippov ϵͳÔÚŸßÓÐÒ»ÌõÇл»Ïß, ÇÒÖÜÆÚ¹ìÓëÇл»ÏßÓÐÁœžö»ò¶àžöœ»µãʱÖÜÆÚ»·Óò·Ö²í³öŒ«ÏÞ»·µÄÇéÐÎ; ¶ÔÓÚ·Ö¶Î\ Hamilton ϵͳµÄÖÜÆÚ»·Óò·Ö²í, Liu ºÍ\ Han ÔÚÎÄÏ×\ \cite{Liu} ÖÐ×öÁËÏàËƵÄÑП¿; µ±ÆœÃæ\ Filippov ϵͳ°üº¬Ò»ÌõÇл»Ïßʱ, Afsharnezhad ºÍ\ Amaleh ÑП¿ÁËϵͳ·¢ÉúµÄÖÜÆÚ¹ì·Ö²íÏÖÏó, ²ÎŒûÎÄÏ×\ \cite{Afsharnezhad}; ¶ÔÓÚ\ \emph{n} ά²»Á¬Ðøϵͳ, Fe\v{c}kan ºÍ\ Pospšª\v{s}il ÔÚÎÄÏ×\ \cite{Feckan2,Feckan3,Feckan4} ÖÐÔËÓ÷ºº¯·ÖÎöÖÐ\ Lyapunov-Schmidt ÔŒ»¯µÄ·œ·šÑП¿ÁËÆäÖÜÆÚ¹ì·Ö²íºÍ»¬¶¯ÖÜÆÚ¹ìµÄ·Ö²íÏÖÏó. Hu ºÍ\ DuÔÚÎÄÏ×\ \cite{Hu} ÖÐÑП¿ÁËÒ»ÀàŸßÓÐÓÐÏÞ¶àÌõÇл»ÏßµÄÆœÃæ²»Á¬ÐøϵͳµÄÖÜÆÚ¹ì·Ö²íÏÖÏó. µÚÈýÖÖ¹¹Ôì·Ö¶Î¹â»¬¶¯ÁŠÏµÍ³Œ«ÏÞ»·µÄ·œ·šÊÇÀûÓÃͬËÞ·Ö²íÀŽ²úÉú¶àžöŒ«ÏÞ»·, Liang, Han ºÍ\ Romanovski ÔÚÎÄÏ×\ \cite{Liang1,Liang2} ÖÐÑП¿ÁËÒ»Àà·Ö¶Î\ Hamilton ϵͳÔÚÈŶ¯ºó, ÓÉͬËÞ¹ì·Ö²í²úÉú³öŒ«ÏÞ»·µÄÏÖÏó.

»ìãç(chaos)ÏÖÏóÎÞÂÛÊǶԹ⻬¶¯ÁŠÏµÍ³»¹ÊǶԷֶι⻬¶¯ÁŠÏµÍ³¶ŒÊÇÒ»Öַdz£ÆÕ±éºÍÖØÒªµÄ·ÇÏßÐÔÏÖÏó, Ôڷֶι⻬ϵͳÖÐ, ²Á±ß¡¢±ßœçÅöײ¡¢»¬¶¯¡¢Õð²üºÍÕ³ÌùµÈÏÖÏ󶌻ᵌÖ»ìãç, ÕâЩ¶ŒÊÇÓÉÓÚÏòÁ¿³¡µÄ²»Á¬ÐøÐÔËùµŒÖµÄ, ²ÎŒûÎÄÏ×\;\cite{Dankowicz2, DemeioL, Ing, Kryzhevich, NusseYorke}. ͬËÞ·Ö²íÊǺܶà¹â»¬¶¯ÁŠÏµÍ³ÍšÏò»ìãçµÄÒ»žöÖØҪ͟Ÿ¶, ¶ø\ Melnikov ·œ·šÊÇ·ÖÎöͬËÞ·Ö²íµÄÓÐÁŠ¹€Ÿß\;\cite{Feckan1, Gruendler, Guck, Meln, Wiggins3}. ŒøÓÚ\ Melnikov ·œ·šÔڹ⻬¶¯ÁŠÏµÍ³ÖеijɹŠÔËÓÃ, Ò»žö×ÔÈ»µÄÎÊÌâŸÍÊÇ¿É·ñœ«ÆäÍƹ㵜·Ö¶Î¹â»¬¶¯ÁŠÏµÍ³ÉÏÀŽ·ÖÎöÆäÖеķֲíÏÖÏó\ ? ¹ØÓÚÕâžöÎÊÌâ, ÒÑŸ­ÓкܶàѧÕßÈ¡µÃÁË·á˶µÄ³É¹û, ²ÎŒûÎÄÏ×\;\cite{Battelli1, Battelli6, Carmona, DuZhang, Granados}. ÕâЩ¹€×÷¶ŒŒÙÉèϵͳÔÚÎŽÈŶ¯Ç°ÖÜÆÚ¹ì»òͬËÞ¹ìÓë²»Á¬ÐøÃæÊǺáœØÏàœ»µÄ. Ò»žöžüÖØÒª, žüÓÐÈ€, Ò²žüÀ§ÄѵÄÇéÐΟÍÊǵ±ÎŽÈŶ¯µÄͬËÞ¹ìµÀÓÐÒ»²¿·ÖÓë²»Á¬ÐøÃæÖغϻòÏàÇÐʱ, ŒŽ»¬¶¯Í¬ËÞ·Ö²í»ò²Á±ßͬËÞ·Ö²íÎÊÌâ, ¶ÔŽËœüÄêÀŽÒ²Óв»ÉÙѧÕßœøÐÐÁËÑП¿. ¶Ô»¬¶¯Í¬ËÞ·Ö²í, Battelli, Fe\v{c}kan, Awrejcewicz, Holicke ºÍ \;Olejnik œ«ÆäÍƹ㵜ÁËÒ»°ãµÄ \;$n$ ά·Ö¶Î¹â»¬¶¯ÁŠÏµÍ³, ²¢ÇÒÊ׎ÎÑÏžñÖ€Ã÷ÁËÔÚŽËÇéÐÎÏÂϵͳҲ»á³öÏÖÏàÓŠµÄ\ Smale ÂíÌã»ìãç, ²ÎŒûÎÄÏ×\;\cite{Awrejcewicz2, Battelli2, Battelli3, Battelli4, Feckan1}; ¶Ô²Á±ßͬËÞ·Ö²í, Du, Li, Shen ºÍ\ Zhang ÑП¿ÁËÔÚÖÜÆÚÍâÁŠÈŶ¯ÏµğßÓÐÁœžöžÕÐÔÔŒÊøÃæµÄ·ÇÏßÐÔµ¹Öõ¥°ÚµÄ²Á±ßͬËÞ·Ö²íºÍ»ìãçÏÖÏó, ²ÎŒûÎÄÏ× \;\cite{DuLiShen,ShenDu}.

ÔÚʵŒÊÓŠÓÃÖÐ, ¶¯ÁŠÏµÍ³ËùÊܵÄÖÜÆÚÍâÁŠÍùÍù²»Ö¹Ò»žö, ÒòŽËÄâÖÜÆÚ»òžÅÖÜÆÚŒ€ÀøϵĶ¯ÁŠÏµÍ³ÓëʵŒÊÎÊÌâÎǺϵÞüºÃ. ¶ÔÓڹ⻬¶¯ÁŠÏµÍ³ÔÚÄâÖÜÆÚÍâÁŠÈŶ¯ÏµÄÑП¿ÔÚÒÔÍùŒžÊ®ÄêŒäÒÑŸ­Ê®·Ö·áž», ²ÎŒûÎÄÏ×\;\cite{Wiggins1, Wiggins2, Wiggins3, Yagasaki}. ÌرðÊǶÔÓÚÔÚÄâÖÜÆÚÈŶ¯ÏµĹ⻬¶¯ÁŠÏµÍ³µÄ»ìãçºÍͬËÞ·Ö²íÏÖÏó, Ide ºÍ \;Wiggins ÓŠÓà \;Melnikov ·œ·šœøÐÐÁËÉîÈëµÄ·ÖÎö, ²ÎŒûÎÄÏ× \;\cite{Wiggins1, Wiggins2, Wiggins3}. Ïà±È¶øÑÔ, ¹ØÓÚÔÚÄâÖÜÆÚÈŶ¯Ïµķֶι⻬¶¯ÁŠÏµÍ³µÄͬËÞ·Ö²íÏÖÏóµÄÑП¿œøչȎʮ·Ö»ºÂý, ÆäÖÐŽú±íÐԵĹ€×÷ÓÐ\ Avramov ºÍ \;Awrejcewicz Ëù×öµÄ¹ØÓÚÄŠ²ÁÕñ×ÓÔÚÄâÖÜÆÚÈŶ¯ÏµÄͬËÞ·Ö²íºÍ»ìãçÏÖÏóµÄÑП¿, ËûÃÇʹÓöà³ß¶È·œ·š(multiple scales) ÇóµÃÁËÏàÓŠµÄµ÷ÖÆ·œ³Ì(modulation equation), ÔÙœáºÏ\;Melnikov ·œ·š¶Ôµ÷ÖÆ·œ³ÌœøÐÐÁË·ÖÎö\ \cite{Avramov}; »¹ÓÐ\ Battelli ºÍ \;Fe\v{c}kan, Du ºÍ\ Gao ·Ö±ðÑП¿ÁËÔÚÖÜÆÚ»òÄâÖÜÆÚÈŶ¯Ïµķֶι⻬¶¯ÁŠÏµÍ³µÄͬËÞ·Ö²íºÍ»ìãçÏÖÏó, ²ÎŒûÎÄÏ×\ \cite{Battelli1, Battelli2, Battelli3, Battelli4, Battelli6, Feckan1, Gao}.

²Á±ß·Ö²íÊǷֶι⻬¶¯ÁŠÏµÍ³ÖÐÓɷǹ⻬ÐÔÓÕµŒµÄÒ»ÀàÖØÒªµÄ·Ö²íÏÖÏó, ÈËÃǶԲÁ±ß·Ö²íÔÚ¹ýÈ¥¶þÊ®ÄêœøÐÐÁËÉîÈëµÄÑП¿. ÑП¿²Á±ß·Ö²íµÄÒ»žöµäÐÍ·œ·šÊÇÓÉ\ Nordmark ËùœšÁ¢\ PoincaršŠ œØÃæ²»Á¬ÐøÓ³ÉäµÄ·œ·š, ËûÊ×ÏÈÓÞ÷œ·šÀŽÑП¿ÅöײϵͳµÄ²Á±ß·Ö²íÏÖÏó, ÓɲÁÅöµãžœœüµÄPoincaršŠœØÃæ²»Á¬ÐøÓ³ÉäµÃ³ö²Á±ß·Ö²íµÄÕý¹æÐÎ(normal form), ÔÙœáºÏµÃµœµÄÕý¹æÐÎÀŽÑП¿·Ö¶Î¹â»¬¶¯ÁŠÏµÍ³ÔÚ²Á±ßµãžœœüµÄ¶¯ÁŠÑ§ÐÐΪ, ²ÎŒûÎÄÏ×\ \cite{Nordmark1}, Ëæºó\ Dankowicz ºÍ\ Nordmark ÔÚÎÄÏ×\ \cite{Dankowicz2} ÖÐœ«ÉÏÊö·œ·šÍƹ㵜һ°ã·Ö¶Î¹â»¬¶¯ÁŠÏµÍ³ÖÐ.

¹ØÓڷֶι⻬¶¯ÁŠÏµÍ³µÄ·Ö²íÎÊÌâµÄÑП¿Ÿ¡¹ÜÏÖÔÚÒÑŸ­Óкܶà³É¹û, µ«ÒÀÈ»ŽæÔںܶàÎÊÌâ. ÓÉÓÚϵͳµÄ²»¿É΢ÐÔËùµŒÖµÄϵͳ¶¯ÁŠÑ§ÐÐΪµÄÆæÒìÐÔ, ʹµÃºÜ¶à»ù±ŸÎÊÌⶌ»¹ÎŽÄÜœâŸö. ŒŽÊ¹¶Ô·Ö¶Î¹â»¬ÏµÍ³ÖÐ×îŒòµ¥µÄ·Ö¶ÎÏßÐÔϵͳ, Æ䶯ÁŠÑ§ÐÐΪ¶ŒÊÇÒì³£·áž»µÄ, »¹ÓкܶàÎÊÌâûÓÐÍêÈ«œâŸö. ³ýŽËÖ®Íâ, Žó¶àÊý·Ö¶Î¹â»¬ÏµÍ³µÄÑП¿ŸÖÏÞÓÚÓÃÊýֵģÄâÀŽ·ÖÎö·Ö²íºÍ»ìãçÏÖÏó; ÓÉÓڷֶι⻬¶¯ÁŠÏµÍ³µÄ»ùŽ¡ÀíÂÛ²»¹»ÍêÕû, ʹµÃžßά»òÎÞÇîάµÄ·Ö¶Î¹â»¬¶¯ÁŠÏµÍ³µÄÑП¿ÄÑÒÔÉîÈë, µÈµÈ. ¶ÔÓÚÕâЩÀ§ÄÑÈÔÐèÈËÃdzÖÐø²»¶ÏµØŬÁŠÈ¥¿Ë·þ.

\section{\kaishu ·Ö¶Î¹â»¬ÏµÍ³µÄÕð²üÏÖÏó}

Õð²üÊÇÁíÒ»Àà·Ö¶Î¹â»¬ÏµÍ³ËùÌØÓеÄÏÖÏó, ¶Ô·Ö¶Î¹â»¬ÏµÍ³µÄÕð²üÏÖÏóµÄÑП¿ÒѳÉΪœüÄêÀŽµÄÒ»žöÈȵã. ÓÈÆäÊÇÔÚ»¬Ä£¿ØÖÆ(sliding-mode control)·œÃæ, Õâ·œÃæµÄÎÄÏ׺ܶà, ŒûךÖø\ \cite{BartoliniF,BartoliniP} Œ°ÆäÒýÎÄ, ÕâЩÎÄÏ×°üÀšÁËÓйطֶι⻬¶¯ÁŠÏµÍ³µÄÕð²üÑП¿µÄ×îÐÂœøÕ¹ºÍ¹«¿ªÎÊÌâ. Budd ºÍ\ Dux ÔÚÎÄÏ×\ \cite{BuddD} ÖжÔÒ»ÀàÔÚÖÜÆÚÍâÁŠÈŶ¯Ïµĵ¥×ÔÓɶȵÄÅöײÕñ×ÓµÄÕð²üÏÖÏóœøÐÐÁËÑП¿, Levant ¶Ô»¬Ä£¿ØÖƵÄÕð²üÏÖÏóÔÚÎÄÏ×\ \cite{Levant} ÖÐ×öÁËÉîÈëµÄÑП¿. Demeio ºÍ\ Lenci ¶ÔŸßÓÐË«²àžÕÐÔÔŒÊøÃæµÄÖÜÆÚÐÔÊÜÆȵ¹Öõ¥°ÚµÄÕð²üÏÖÏóÔÚÎÄÏ×\ \cite{DemeioL} ÖÐœøÐÐÁËÑП¿. Drossel ºÍ\ Prellberg ¶ÔÒ»žöÖʵãÔÚÖÜÆÚÐÔÍâÁŠÈŶ¯ÏµĺÐ×ÓÄÚµÄÕð²üÏÖÏóÔÚÎÄÏ×\ \cite{Drossel} ÖÐœøÐÐÁËÑП¿.

\section{\kaishu ±ŸÎĵÄÖ÷Òª¹€×÷}

±ŸÎÄœ«ÑП¿ÎÞ×èÄᵯÌøÇòÄ£Ð͵ÄÍêÕûÕð²üÏÖÏó. µ¯ÌøÇòÄ£ÐÍÊÇÒ»žöŸ­µäµÄÅöײģÐÍ, ×îÔçÓÉ\ Fermi Ìá³ö\ \cite{Fermi}, ÆäÎïÀíÄ£ÐÍÈçÍŒ\ 1 ËùÊŸ, ËüÓÉÒ»žöµ¯ÌøÇòºÍ×öÖÜÆÚÐÔÕýÏÒÕñ¶¯µÄÆœ°å×é³É. µ¯ÌøÇòÄ£ÐÍÔÚÎïÀíѧÉÏÓзdz£¹ã·ºµÄÓŠÓÃ, Ò»Ö±Œ€·¢×ÅѧÕ߶ÔÆäÉîÈëÑП¿µÄÐËÈ€. žÃÄ£ÐÍÐÎÊœËäÈ»Œòµ¥, µ«¶¯ÁŠÑ§ÐÐΪȎʮ·Ö·áž»\ \cite{Tufillaro,Cristina,Vogel}, ±ÈÈçÓÉÓÚÅöײ¿É·Öµ¯ÐÔÅöײ¡¢·Çµ¯ÐÔÅöײ¡¢ÍêÈ«·Çµ¯ÐÔÅöײ, ÿÖÖÅöײÐÎÊœ¶ŒÓО÷×ÔµÄÌصã.  ÑП¿±íÃ÷, ºÍŽó¶àÊý·Ö¶Î¹â»¬ÏµÍ³Ò»Ñù, Ëü³ýÁËŸßÓÐÓ럭µä¹â»¬¶¯ÁŠÏµÍ³ÀàËƵķֲíÏÖÏó, Èç°°œáµã·Ö²í¡¢±¶ÖÜÆÚ·Ö²íÍâ, »¹ÓкܶàÓɷǹ⻬ÐÔÓÕµŒµÄÌØÊâ·Ö²íÏÖÏó, Èç²ÁÅö¡¢±ßœçÅöײºÍÕð²üµÈ.

¹ØÓÚµ¯ÌøÇòÄ£Ð͵ÄÕð²üÏÖÏó, ÔÚÎÄÏ×\ \cite{Luck} ÖÐ, ×îÔçÓÉ\ Luck ºÍ\ Mehta ¿ªÊŒÑП¿, µ«Ö»ÔÚµ¯ÐÔϵÊýÇ÷ÓÚ\ 0 ºÍ \ 1 ʱžøÁˎֲڵĹÀŒÆ, ÔÚËæºóµÄ¹ØÓÚµ¯ÌøÇòÄ£ÐÍÕð²üµÄÑП¿ÖÐ\ \cite{Vogel,Barroso} Ö÷ÒªÌåÏÖÔÚÊýֵģÄâÉÏ, µ¯ÌøÇòÄ£ÐÍ·¢ÉúÍêÕûÕð²üµÄÌõŒþһֱΎžø³ö.


±ŸÎÄœ«ÌÖÂÛÎÞ×èÄᵯÌøÇòÄ£Ð͵ÄÍêÕûÕð²üÏÖÏó, ŒŽÔÚÓÐÏÞµÄʱŒäÄÚ, µ¯ÌøÇòÓë×öÖÜÆÚÐÔÕñ¶¯µÄÆœ°å·¢ÉúÎÞÊýŽÎÅöײ²¢Ç÷ÓÚŸ²Ö¹µÄÏÖÏó. ÀûÓÃÖÐÖµ¶šÀí, žø³öÁËÎÞ×èÄᵯÌøÇòÄ£ÐÍÔÚ·Žµ¯ÏµÊý·Ö±ðÇ÷ÓÚ\ $0$ ºÍÇ÷ÓÚ\ $1$ ʱ·¢ÉúÍêÕûÕð²üʱ, ϵͳ²ÎÊýÓë³õʌֵËùÂú×ãµÄ³ä·ÖÌõŒþ. ²¢œøÐÐÁËÏàÓŠµÄÊýֵģÄâ, ÊýֵģÄâµÄœá¹ûºÍÎÒÃǵÄÀíÂÛÔ€²âÍêÈ«ÎǺÏ. ͬʱÔÚµ¯ÌøÇòÄ£ÐÍ·¢ÉúÖÜÆÚÐÔÕð²üµÄʱºò, µ±ÖÜÆÚÐÔÕð²üµÄÌõŒþœÓœüÁÙœç׎̬ʱ, ÀûÓÃ\ Taylor ¹«Êœžø³öÁËÕð²üʱŒäÓëϵͳ²ÎÊýŒäµÄœüËƹÀŒÆÊœ.
%%%%%%%%%%%%%%%%%%%%%%%%%%%%%%%%%%%%%%%%%%%%%%%%%%%%%%%%%%%%%%%%%%%%%%%%%%%%%%%%
\chapter{\heiti ÎÞ×èÄᵯÌøÇòÄ£Ð͵ÄÍêÕûÕð²ü}
\section{\kaishu µ¯ÌøÇòµÄÊýѧģÐÍ}

±ŸÎÄ¿ŒÂǵÄÎÞ×èÄᵯÌøÇòÄ£ÐÍÓÉÒ»žöµ¯ÌøÇòºÍ×öÖÜÆÚÐÔÕýÏÒÕñ¶¯µÄÆœ°å×é³É, Ϊ·œ±ãÌÖÂÛ, ŒÙ¶šÆœ°åµÄ³õÊŒÏàλ\ $\varphi$ Ϊ\ 0, ÔòÆœ°åµÄÔ˶¯·œ³ÌΪ
 \begin{equation}\label{pd}
 S(t)=A\sin(\omega t),
\end{equation}
ÆäÖÐ\ \emph{A} ΪƜ°åµÄÕñ·ù, $\omega$ ΪƜ°åµÄÕñ¶¯ÆµÂÊ. ŒÙÉ赯ÌøÇòµÄÖÊÁ¿Ïà¶ÔÆœ°å¿ÉÒÔºöÂÔ²»ŒÆ, ²¢ÇÒµ¯ÌøÇòÓëÆœ°å·¢ÉúµÄÅöײΪµ¯ÐÔÅöײ, Éè·Žµ¯ÏµÊýΪ\ \emph{r}, ÆäÖÐ
$0<r<1$, Èç¹û\ $r=0$, ÔòÅöײΪÍêÈ«·Çµ¯ÐÔÅöײ.

ÏÂÃæÎÒÃÇÀŽœéÉܵ¯ÌøÇòÄ£ÐÍÎÞÁ¿žÙ»¯ÏµÍ³µÄÍƵŒ¹ý³Ì.

Ê×ÏÈœøÐÐʱŒä³ß¶È±ä»», ȡƜ̚µÄÕñ¶¯ÖÜÆÚ\ $\frac{2\pi}{\omega}$ ΪÌØÕ÷ʱŒä, ÔòÎÞÁ¿žÙʱŒäΪ
\begin{equation}\label{pdtime}
  \tau=\frac{\omega}{2\pi}t.
\end{equation}
ȡϵͳµÄÌØÕ÷ŒÓËÙ¶ÈΪ\ $\frac{g}{2}$, ÆäÖÐ\ g ΪÖØÁŠŒÓËÙ¶È, ÔòÎÞÁ¿žÙŒÓËÙ¶ÈΪ
\begin{equation}\label{pda}
  \tilde{g}=\frac{g}{\frac{g}{2}}=2.
\end{equation}
ÔòÓÉ\ (\ref{pdtime}) ºÍ\ (\ref{pda}) ¿ÉµÃÎÞÁ¿žÙËÙ¶ÈΪ
\begin{equation}\label{pdv}
  \tilde{v}=\tau\tilde{g}=\frac{2t\omega}{2\pi}=\frac{t\omega}{\pi}=\frac{tg}{\frac{g\pi}{\omega}},
\end{equation}
ŒŽÏµÍ³µÄÌØÕ÷ËÙ¶ÈΪ\ $\frac{g\pi}{\omega}$.
ÔòÓÉ\ (\ref{pdtime}) ºÍ\ (\ref{pdv}) ¿ÉµÃÎÞÁ¿žÙλÒÆ×ø±êΪ
\begin{equation}\label{pdl}
  \tilde{S}=\tilde{v}\tau=\frac{\omega^2}{2\pi^2}t^2=\frac{\frac{1}{2}gt^2}{\frac{2g\pi^2}{\omega^2}},
\end{equation}
ŒŽÏµÍ³µÄÌØÕ÷λÒÆ×ø±êΪ\ $\frac{2g\pi^2}{\omega^2}$.

ÁªÁ¢(\ref{pd})¡¢(\ref{pdtime})¡¢(\ref{pdl})µÃÎÞÁ¿žÙ»¯ºóÆœ°åµÄÔ˶¯·œ³ÌΪ
\begin{equation}\label{pds}
  S(\tau)=\frac{A\omega^2}{2g\pi^2}\sin(2\pi\tau).
\end{equation}
ÒýÈë\ $\Gamma=\frac{A\omega^2}{\pi g}$, Ôò\ $\Gamma$ ΪÎÞÁ¿žÙ»¯ÒÔºóÆœ°åµÄÕñ¶¯ŒÓËٶȵÄ×îŽóÖµ, ×îÖÕÆœ°åµÄÔ˶¯·œ³ÌΪ
\begin{equation}\label{pdf}
  S(\tau)=\frac{\Gamma}{2\pi}\sin(2\pi\tau).
\end{equation}
ÔòœøÐÐÎÞÁ¿žÖ»¯ÒÔºó, µ¯ÌøÇòÄ£Ð͵Äϵͳ²ÎÊýÖ»ÓÐ\ $\Gamma$ ºÍµ¯ÌøÇòÓëÆœ°åÅöײºóµÄ·Žµ¯ÏµÊý\ \emph{r}.

ÏÂÃæÎÒÃÇÀŽœéÉܵ¯ÌøÇòÄ£Ð͵ĶþάӳÉäµÄÍƵŒ¹ý³Ì.

ÁÌøÇòÎÞÁ¿žÙ»¯ºóµÄλÒƺ¯ÊýΪ\ $H(\tau)$, ³õʌλÖÃΪ\ $H_0$, $V_0$ Ϊ³õÊŒËÙ¶È, ÓëÆœ°å·ÖÀëµÄ³õʌʱŒäΪ\ $\tau_0$, ŒŽÓÐ\ $H_0=S(\tau_0)$, Ôò¿ÉµÃ
\begin{equation}\label{ballh}
  H(\tau)=H_0+V_0(\tau-\tau_0)-(\tau-\tau_0)^2.
\end{equation}
Éè\ $W_0$ Ϊµ¯ÌøÇòÓëÆœ°åµÚÒ»ŽÎ·ÖÀëʱµÄÏà¶ÔËÙ¶È, ÔòÔÚ\ $\tau_0$ ʱÓУº
\begin{equation}\label{ballw}
  V_0=W_0+S'(\tau_0)=W_0+\Gamma\cos(2\pi\tau_0).
\end{equation}

É赯ÌøÇòÓëÆœ°åÖ®ŒäµÄÏà¶ÔŸàÀëΪ\ $X(\tau)$, ŒŽÓÐ
\begin{equation}\label{ballH}
  X(\tau)=H(\tau)-S(\tau),
\end{equation}
ŒÇµ¯ÌøÇòÓëÆœ°åµÚÒ»ŽÎÅöײµÄʱŒäΪ\ $\tau_1$, Ôò\ $\tau_1$ Ϊ·œ³Ì
\begin{center}
$X(\tau)=0$
\end{center}
µÄœâ, ÇÒÊÇÂú×ã²»µÈÊœ\ $\tau_1>\tau_0$ µÄ×îСœâ.
ÔòÁªÁ¢\ (\ref{pdf})¡¢(\ref{ballh})¡¢(\ref{ballw})¡¢(\ref{ballH}) ¿ÉµÃ\ $\tau_1$ Óë\ $\tau_0$ µÄ¹Øϵʜ
\begin{equation}\label{tau0}
\frac{\Gamma}{2\pi}\big[\sin(2\pi\tau_{0})-\sin(2\pi\tau_{1})\big]+\big[W_{0}+\Gamma\cos(2\pi\tau_{0})\big](\tau_{1}-\tau_{0})-(\tau_{1}-\tau_{0})^{2}=0,
\end{equation}
ÓÉ\ (\ref{tau0}) ¿ÉµÃµ¯ÌøÇòÓëÆœ°åÔÚÅöײʱ¿Ì\ $\tau_1$ Ç°µÄÏà¶ÔËÙ¶ÈΪ
\begin{equation}\label{w0q}
  W^{(-)}_1=X'(\tau_1)=W_0-2(\tau_1-\tau_0)+\Gamma[\cos(2\pi\tau_0)-\cos(2\pi\tau_1)],
\end{equation}
ŽËʱÏà¶ÔËÙ¶ÈΪžºÖµ, ÓÉÓÚÊǵ¯ÐÔÅöײ, ÔÚÅöײʱ¿Ì\ $\tau_1$ ºóµÄÏà¶Ô˲ʱËÙ¶ÈΪ
\begin{equation}\label{w0h}
   W_1=-rW^{(-)}_1,
\end{equation}
ŽËʱÏà¶ÔËÙ¶ÈΪÕýÖµ, ÁªÁ¢\ (\ref{w0q}) ºÍ\ (\ref{w0h}) ¿ÉµÃ
\begin{equation}\label{w0}
   W_{1}=-rW_{0}+2r(\tau_{1}-\tau_{0})-r\Gamma\big[\cos(2\pi\tau_{0})-\cos(2\pi\tau_{1})\big].
\end{equation}
Ôòµ¯ÌøÇòÄ£ÐÍ¿ÉÓöþάӳÉä\ $(\tau_n,W_n)\mapsto (\tau_{n+1},W_{n+1})$ ÀŽÃèÊö, ÆäÖÐ\ $(\tau_n,W_n)$ ºÍ\ $(\tau_{n+1},W_{n+1})$ Âú×ãÏÂÁзœ³Ì×飺
\begin{multline}\label{tau}
 \frac{\Gamma}{2\pi}\big[\sin(2\pi\tau_{n})-\sin(2\pi\tau_{n+1})\big]\\
 +\big[W_{n}+\Gamma\cos(2\pi\tau_{n})\big](\tau_{n+1}-\tau_{n})-(\tau_{n+1}-\tau_{n})^{2}=0,
\end{multline}
\begin{equation}\label{W}
  W_{n+1}=-rW_{n}+2r(\tau_{n+1}-\tau_{n})-r\Gamma\big[\cos(2\pi\tau_{n})-\cos(2\pi\tau_{n+1})\big],
\end{equation}
ÆäÖÐ\ $\tau_n$ ΪµÚ\ \emph{n} ŽÎÅöײʱµÄʱŒä,\ $W_n$ ΪµÚ\ \emph{n} ŽÎÅöײʱµÄµ¯ÌøÇòÏà¶ÔÓÚÆœ°åµÄÏà¶ÔËÙ¶È. ÏÔȻϵͳ\ (\ref{tau}-\ref{W}) µÄ¹ìµÀÓɳõʌֵ\ $(\tau_0,W_0)$ È·¶š.
\section{\kaishu ±ŸÎĵÄÖ÷Òªœá¹û}

±ŸÎÄÌÖÂÛÔÚ\ \emph{r} ºÍ\ $1-r$ ³ä·ÖСµÄÁœÖÖÇé¿öÏÂϵͳ\ (\ref{tau}-\ref{W}) µÄÍêÕûÕð²üÏÖÏó, ŒŽÔÚÓÐÏÞµÄʱŒäÄÚ, µ¯ÌøÇòÓëÆœ°å·¢ÉúÎÞÊýŽÎÅöײ²¢Ç÷ÓÚŸ²Ö¹µÄÏÖÏó. Éè³õʌʱŒäΪ\ $\tau_0$, $\tau_n$ ΪµÚ\ \emph{n} ŽÎÅöײµÄʱ¿Ì, Ôò\ $\Delta_n$:$=\tau_{n+1}-\tau_n$ ΪµÚ\ $n+1$ ŽÎÅöײÓëµÚ\ \emph{n} ŽÎÅöײµÄŒäžôʱŒä, Ôòϵͳ\ (\ref{tau}-\ref{W}) ·¢ÉúÍêÕûÕð²üµÄ³äÒªÌõŒþΪ: ¶Ô\ $n=1, 2,\dots$, $\Delta_n>0$, ÇÒ:
 \begin{equation}\label{Ctime}
 T=\sum_{n=1}^{\infty}\Delta_n< +\infty.
 \end{equation}
³Æ\ \emph{T} ΪÕð²üʱŒä.

±ŸÎĹØÓÚϵͳ\ (\ref{tau}-\ref{W}) ·¢ÉúÍêÕûÕð²üµÄÖ÷Òªœá¹ûÊÇ:
\begin{thm}\label{T1}
{\upshape\kaishu ~~Éè \ $\pi\Gamma>1$, $0<r<\frac{1}{13}$, Ôòµ±\ $\tau_0\in[0,\tau^0_*)$, \ $W_0\leq \frac{1}{7}(1-13r)(\tau^0_*-\tau_0)$,ÆäÖÐ\ $\tau^0_*=\frac{1}{2\pi}\arcsin\frac{5}{7\pi\Gamma}$, Ó³Éä\ (\ref{tau}-\ref{W}) È¡³õֵΪ\ $(\tau_0,W_0)$ ʱ·¢ÉúÍêÕûÕð²ü.}
\end{thm}

\begin{thm}\label{T2}
{\upshape\kaishu ~~Éè\ $\pi\Gamma>1$, $\frac{3}{5}<r<1$, Ôòµ±\ $\tau_0\in[0,\frac{1}{2\pi}\arcsin\frac{1-r}{2r\pi\Gamma}]$, \ $W_0\leq (1-r)^2(\tau^1_*-\tau_0)$, ÆäÖÐ\ $\tau^1_*=\frac{1}{2\pi}\arcsin\frac{5(1-r)}{6\pi\Gamma}$ ʱ, Ó³Éä\ (\ref{tau}-\ref{W}) È¡³õֵΪ\ $(\tau_0,W_0)$ ʱ·¢ÉúÍêÕûÕð²ü.}
\end{thm}

µ¯ÌøÇòÄ£Ð͵ÄÖÜÆÚÐÔÕð²üÊÇÖžµ¯ÌøÇòŸ²Ö¹ÔÚÆœ°åÖ®ºó, ÔÚÆœ°åµÄÏÂÒ»žöÕñ¶¯ÖÜÆÚʱÓëÆœ°å·¢Éú·ÖÀë, ·¢ÉúÍêÕûÕð²ü, ÓÖŸ²Ö¹ÔÚÆœ°åÉϵÄÖÜÆÚÔ˶¯. ÖÜÆÚÐÔÕð²ü·¢ÉúµÄÒ»žö±ØÒªÌõŒþÊÇ:
\begin{equation}\label{22}
 \pi\Gamma\sin(2\pi\tau)>1.
\end{equation}

¹ØÓÚµ¯ÌøÇòÄ£ÐÍ·¢ÉúÖÜÆÚÐÔÕð²üµÄÕð²üʱŒä, ÎÒÃÇÓÐÈçϹÀŒÆ:
\begin{thm}\label{T3}
{\upshape\kaishu ~~µ¯ÌøÇòÄ£ÐÍ\ (\ref{tau}-\ref{W}) ÔÚ·¢ÉúÖÜÆÚÐÔÕð²üʱ, µ±\ $\pi\Gamma\rightarrow1$ ʱ, Õð²üʱŒä\ \emph{T} Óëϵͳ²ÎÊýµÄ¹ØϵΪ
\begin{center}
  $T\cong\frac{\lambda(r)}{\pi}\sqrt{\pi\Gamma-1}$,
\end{center}
ÆäÖÐ\ $\lambda(r)$ Ϊ·Žµ¯ÏµÊý\ \emph{r} µÄº¯Êý, ÆäŸßÌå±íŽïÊœÔÚÖ€Ã÷¹ý³ÌÖОø³ö.}
\end{thm}

\chapter{\heiti Ö÷Òªœá¹ûµÄÖ€Ã÷}
\section{\kaishu Ïà¹ØÒýÀí}

ÔÚÖ€Ã÷¶šÀí\ $\ref{T1}$ ºÍ\ $\ref{T2}$ ֮ǰ, ÎÒÃÇÊ×ÏÈÖ€Ã÷ÏÂÃæµÄÒýÀí.
\begin{lem}
{\upshape\kaishu ~~Éè \ \emph{k} ΪÕýÕûÊý, $x$¡¢$y$ Âú×ã \ $2k\pi<y<x<2k\pi+\frac{\pi}{2}$, ÔòÓÐ
\begin{equation}\label{L1}
  \frac{\sin x-\sin y}{x-y}-\cos\Big(\frac{3}{5}x+\frac{2}{5}y\Big)>0.
\end{equation}
}\label{lemma}
\end{lem}

{\heiti \textbf{Ö€Ã÷} }~~ Áî \ $\Delta y=x-y$, ÓÉŒÙÉèÖª\ $\Delta y\in(0,\frac{\pi}{2})$, ÓÉŽøÓÐ \ Lagrange ÓàÏîµÄ \ Taylor ¹«Êœ, ŽæÔÚ\ $\xi_1$¡¢$\xi_2\in(0,\frac{\pi}{2})$ ʹµÃ
  \begin{equation}
    \sin x-\sin y=\Delta y\cos y-\frac{(\Delta y)^2}{2}\sin y-\frac{(\Delta y)^3}{6}\cos y+\frac{(\Delta y)^4}{24}\sin\xi_1,
  \end{equation}
  \begin{eqnarray*}
  % \nonumber to remove numbering (before each equation)
  \cos\Big(\frac{3}{5}x+\frac{2}{5}y\Big) &=& \cos y-\frac{3\Delta y}{5}\sin y-\frac{9(\Delta y)^2}{50}\cos y+\frac{9(\Delta y)^3}{250}\sin y \\
     &+&     \frac{27(\Delta y)^4}{5000}\cos y-\frac{81(\Delta y)^5}{125000}\sin\xi_2.
  \end{eqnarray*}
ÓÉ \ $\frac{(\Delta y)^4}{24}\sin\xi_1>0$, $-\frac{81(\Delta y)^5}{125000}\sin\xi_2<0$, ¿ÉµÃ
\begin{equation}
  \frac{\sin x-\sin y}{\Delta y}>\cos y-\frac{\Delta y}{2}\sin y-\frac{(\Delta y)^2}{6}\cos y,
 \end{equation}

\begin{equation}
\cos\Big(\frac{3x}{5}+\frac{2y}{5}\Big)<\cos y-\frac{3\Delta y}{5}\sin y-\frac{9(\Delta y)^2}{50}\cos y+\frac{9(\Delta y)^3}{250}\sin y+\frac{27(\Delta y)^4}{5000}\cos y.
\end{equation}
 ÔòËùÒªÖ€µÄ\ (\ref{L1}) ÊœµÈŒÛÓÚ:
\begin{equation}\label{L01}
  \frac{\Delta y}{2}\sin y+\frac{(\Delta y)^2}{15}\cos y>\frac{9(\Delta y)^3}{50}\sin y+\frac{27(\Delta y)^4}{1000}\cos y.
\end{equation}

¶ø\ (\ref{L01}) ÓֵȌÛÓÚ
\begin{equation}\label{L001}
  \frac{\Delta y}{2}\sin y+\frac{(\Delta y)^2}{15}\cos y = \frac{200}{81}\Big(\frac{81\Delta y}{400}\sin y+\frac{27(\Delta y)^2}{1000}\cos y\Big),
\end{equation}
µ± \ $2\pi<y<x<2\pi+\frac{\pi}{2}$ ʱ, ÓÐ \ $(\Delta y)^2<\frac{\pi^2}{4}<\frac{(3.142)^2}{4}<\frac{200}{81}$,
ÒòŽË\ (\ref{L001}) µÈŒÛÓÚ
\begin{equation}
 \frac{200}{81}\Big(\frac{81\Delta y}{400}\sin y+\frac{27(\Delta y)^2}{1000}\cos y\Big) > (\Delta y)^2 \Big(\frac{81\Delta y}{400}\sin y+\frac{27(\Delta y)^2}{1000}\cos y\Big),
\end{equation}
ÓÖÓÉ \ $\frac{81}{400}>\frac{9}{50}$, ¿ÉµÃ
\begin{equation}\label{L0001}
(\Delta y)^2 \Big(\frac{81\Delta y}{400}\sin y+\frac{27(\Delta y)^2}{1000}\cos y \Big)> \frac{9(\Delta y)^3}{50}\sin y+\frac{27(\Delta y)^4}{1000}\cos y.
\end{equation}
¶ø\ (\ref{L0001}) µÈŒÛÓÚ:
\begin{equation}\label{Lend}
  \frac{9}{400}\Delta y\sin y>0.
\end{equation}
ÓÉÓÚ\ $\Delta y \in(0,\frac{\pi}{2})$, ÒòŽË\ (\ref{Lend}) ³ÉÁ¢, ŽÓ¶ø\ (\ref{L1}) ³ÉÁ¢. Ö€±Ï.
\section{\kaishu ÍêÕûÕð²üµÄÌõŒþ}

ÏÖÔÚÎÒÃÇÀŽÖ€Ã÷±ŸÎĵÄÖ÷Òªœá¹û.

{\heiti \textbf{¶šÀí\ $\ref{T1}$ µÄÖ€Ã÷} }~~Õð²üÔÚ²»³¬¹ý \ $\tau^0_*$ ʱÍê³É, Âú×ãÍêÕûÕð²üµÄÒªÇó, ÔòÓÉ\ (\ref{Ctime}), Õð²üʱŒä\ \emph{T} Âú×ã:
\begin{equation}\label{T1time}
  T\leq\tau_*^0-\tau_0.
\end{equation}

ÏÂÃæÓÉ\ (\ref{T1time}) µÃµœ\ $W_0$ ÐèÒªÂú×ãµÄÌõŒþ. Ê×ÏÈ¿ŒÂǵÚÒ»ŽÎʱŒäŒäžô \ $\Delta_0$, ËüÓÉÏÂʜȷ¶š
  \begin{equation}\label{T104}
  \frac{\Gamma}{2\pi}[\sin(2\pi\tau_0)-\sin(2\pi\tau_0+2\pi\Delta_0)]+[W_1+\Gamma\cos(2\pi\tau_0)]\Delta_0-\Delta_0^{2}=0.
\end{equation}
¶Ô\ (\ref{T104}) ×ó±ßÓÃÒ»ŽÎ \ Lagrange ÖÐÖµ¶šÀí, ÔòÓÐ \ $\xi_{10}\in(\tau_0,\tau_1)$ Âú×ãÏÂÊœ
\begin{equation}\label{T105}
   \Delta_0[-\Gamma\cos(2\pi\xi_{10})+W_0+\Gamma\cos(2\pi\tau_0)-\Delta_0]=0.
\end{equation}
ÓÉÓÚ\ $\Delta_0\neq 0$, ÒòŽËÓÉ\ (\ref{T105}) µÃ
\begin{equation}\label{T106}
  \Gamma\cos(2\pi\xi_{10})-\Gamma\cos(2\pi\tau_0)+\Delta_0=W_0,
\end{equation}
¶Ô\ (\ref{T106}) ×ó±ßÔÙÓÃÒ»ŽÎ \ Lagrange ÖÐÖµ¶šÀíµÃ, ŽæÔÚ\ $\xi_{11}\in(\tau_0,\xi_{10})$, ʹµÃ:
\begin{equation}\label{T107}
  \Delta_0-2\pi\Gamma\sin(2\pi\xi_{11})(\xi_{10}-\tau_0)=W_0.
\end{equation}

ÓÉÒýÀí \;\ref{lemma} µÃ
\begin{equation}
  \frac{\sin(2\pi\tau_1)-\sin(2\pi\tau_0)}{\tau_1-\tau_0}=2\pi\cos(2\pi\xi_{10})>2\pi\cos \Big [2\pi \Big(\frac{3\tau_1}{5}+\frac{2\tau_0}{5}\Big)\Big].
\end{equation}
 ÓÉÓÚ\ $2\pi\cos(2\pi x)$, µ±\ $x\in(0,\tau^0_*)$ ʱµ¥µ÷µÝŒõ, ÒòŽË
 \begin{equation}\label{T108}
   \xi_{10}<\frac{3\tau_1}{5}+\frac{2\tau_0}{5},
 \end{equation}
 ÓÉ\ (\ref{T108}) Áœ±ßŒõÈ¥\ $\tau_0$ µÃ
 \begin{equation}\label{T109}
  \xi_{10}-\tau_0<\frac{3\tau_1}{5}+\frac{2\tau_0}{5}-\tau_0=\frac{3}{5}\Delta_0.
\end{equation}
 ÓÉ\ (\ref{T107}) ¿ÉŒû, µ±\ $2\pi\Gamma\sin(2\pi\xi_{11})(\xi_{10}-\tau_0)$ È¡ÖµÔœŽó, ËùµÃ\ $\Delta_0$ ÔœŽó, ÕâÊÇÒòΪÔÚ\ (\ref{T104}) ÖÐ,\ $W_0$ ÊÇÈ·¶šµÄ, ÊÇÒÑÖª\ $W_0$ ºÍ\ $\tau_0$ ÀŽÇó\ $\Delta_0$ µÄ, ×ó±ßÊÇ\ $[1-\frac{6}{5}\pi\Gamma\sin(2\pi\xi_{11})]\Delta_0$,\ $\frac{6}{5}\pi\Gamma\sin(2\pi\xi_{11})<1$,
ËùÒÔŽËʱ\ $2\pi\Gamma\sin(2\pi\xi_{11})$ È¡ÖµÔœŽó, ËùµÃ \ $\Delta_0$ ÔœŽó.

ÍêÕûÕð²üÖÐ \ $0<\tau_0<\dots<\tau_n<\dots<\tau^0_*<\frac{1}{4}$, ÓÉÓÚ\ $2\pi\sin(2\pi x),x\in(0,\frac{1}{4})$ µ¥µ÷µÝÔö, ÇÒ\ $0<\xi_{11}<\tau^{0}_{*}<\frac{1}{4}$, ËùÒÔ
\begin{equation}\label{T110}
  2\pi\Gamma\sin(2\pi\xi_{11})<2\pi\Gamma\sin(2\pi\tau^0_*)=\frac{10}{7},
\end{equation}
ÓÉ\ (\ref{T109}), (\ref{T110}) µÃ\ $2\pi\Gamma\sin(2\pi\xi_{11})(\xi_{10}-\tau_0)<\frac{6}{7}\Delta_0$,  ŽúÈë\ (\ref{T107}) µÃ
\begin{equation}\label{T111}
  \Delta_0<7W_0.
\end{equation}

È»ºó¿ŒÂǵڶþŽÎʱŒäŒäžô\ $\Delta_1$, ͬÑù¿ÉµÃ
\begin{equation}\label{T112}
  \Delta_1<7W_1,
\end{equation}
 ÇÒ
 \begin{equation}\label{T113}
  W_1= r\{-W_0+2\Delta_0+\Gamma[\cos(2\pi\tau_1)-\cos(2\pi\tau_0)]\} < 13r W_0,
\end{equation}
ÓÉ\ (\ref{T112}), (\ref{T113}) µÃ\ $\Delta_1<7W_1<7W_0 13r$. ÒÔŽËÀàÍƵÃ\ $\Delta_n<7W_0r^{n-1}(13)^{n-1}$.

µ±\ $r<\frac{1}{13}$ ʱ, Õð²üʱŒä\ \emph{T} Ϊ
 \begin{equation}\label{T114}
 T=\sum_{n=1}^{\infty}\Delta_n<\sum_{n=0}^\infty7W_0r^{n}(13)^{n}=\frac{7W_0}{1-13r},
\end{equation}
ÔÙÓÉ\ (\ref{Ctime}) µÃ
\begin{equation}\label{T115}
   \frac{7W_0}{1-13r}\leq\tau^0_*-\tau_0.
   \end{equation}
Ö€±Ï.

{\heiti \textbf{¶šÀí\ $\ref{T2}$ µÄÖ€Ã÷} }~~Õð²üÔÚ²»³¬¹ý \ $\tau^1_*$ ʱÍê³É, Âú×ãÍêÕûÕð²üµÄÒªÇó, µ±\ $\frac{3}{5}<r<1$ ʱ,
$\frac{1}{2\pi}\arcsin\frac{1-r}{2r\pi\Gamma}<\tau^1_*=\frac{1}{2\pi}\arcsin\frac{5(1-r)}{6\pi\Gamma}$, ʹµÃ\ $\tau^1_*-\tau_0>0$, ÔòÓÉ\ (\ref{Ctime}), Õð²üʱŒä\ $\emph{T}$ Âú×ã:
\begin{equation}\label{T201}
 T=\sum_{n=0}^{\infty}\Delta_n\leq\tau^1_*-\tau_0.
 \end{equation}

ÏÂÃæÓÉ\ (\ref{T201}) µÃµœ \ $W_0$ ÐèÒªÂú×ãµÄÌõŒþ.
Ê×ÏÈ¿ŒÂǵÚÒ»ŽÎʱŒäŒäžô \ $\Delta_0=\tau_1-\tau_0$, ËüÓÉÏÂʜȷ¶š
  \begin{equation}\label{T2time}
  \frac{\Gamma}{2\pi}\big[\sin(2\pi\tau_0)-\sin(2\pi\tau_0+2\pi\Delta_0)\big]+\big[W_0+\Gamma\cos(2\pi\tau_0)\big]\Delta_0-\Delta_0^{2}=0,
\end{equation}
ÓÉ\ (\ref{T2time}) Œ° \ Lagrange ÖÐÖµ¶šÀí, ŽæÔÚ \ $\xi_{20}\in(\tau_0,\tau_1)$  ʹµÃ:
\begin{equation}\label{T202}
   \Delta_0\big[-\Gamma\cos(2\pi\xi_{20})+W_0+\Gamma\cos(2\pi\tau_0)-\Delta_0\big]=0,
\end{equation}
ÓÉÓÚ\ $\Delta_0\neq 0$, ÓÉ\ (\ref{T202}) ¿ÉµÃ
\begin{equation}\label{T203}
  \Gamma\cos(2\pi\xi_{20})-\Gamma\cos(2\pi\tau_0)+\Delta_0=W_0.
\end{equation}
ÓÉ\ (\ref{T203}) ºÍ \ Lagrange ÖÐÖµ¶šÀí, ŽæÔÚ\ $\xi_{21}\in(\tau_0,\xi_{20})$, ʹµÃ:
\begin{equation}\label{T204}
  \Delta_0-2\pi\Gamma\sin(2\pi\xi_{21})(\xi_{20}-\tau_0)=W_0.
\end{equation}

ÓÉÒýÀí \;\ref{lemma} µÃ
\begin{equation*}
  \frac{\sin(2\pi\tau_1)-\sin(2\pi\tau_0)}{\tau_1-\tau_0}=2\pi\cos(2\pi\xi_{20})>2\pi\cos\Big[2\pi \Big(\frac{3\tau_1}{5}+\frac{2\tau_0}{5}\Big)\Big],
\end{equation*}
 ÓÉÓÚ\ $2\pi\cos(2\pi x)$, µ±\ $x\in(\tau_0,\tau^1_*)$ ʱµ¥µ÷µÝŒõ,  ËùÒÔ \ $\xi_{20}<\frac{3\tau_1}{5}+\frac{2\tau_0}{5}$, ŒŽ
 \begin{equation}\label{T205}
  \xi_{20}-\tau_0<\frac{3}{5}\Delta_0.
\end{equation}

ÔÚ\ (\ref{T204}) ÖÐ, µ± \ $2\pi\Gamma\sin(2\pi\xi_{21})(\xi_{20}-\tau_0)$ È¡ÖµÔœŽó, ËùµÃ \ $\Delta_0$ ÔœŽó, ÕâÊÇÒòΪÔÚ\ (\ref{T2time}) ÖÐ,
ÊÇÒÑÖª \ $W_0$ ºÍ \ $\tau_0$ ÀŽÇó \ $\Delta_0$ µÄ, ×ó±ßÊÇ \ $[1-\frac{6}{5}\pi\Gamma\sin(2\pi\xi_{21})]\Delta_0$,
$\frac{6}{5}\pi\Gamma\sin(2\pi\xi_{21})<1$,
ËùÒÔŽËʱ \ $2\pi\Gamma\sin(2\pi\xi_{21})$ È¡ÖµÔœŽó, ËùµÃ \ $\Delta_0$ ÔœŽó.

ÍêÕûÕð²üÖÐ \ $0<\tau_0<\dots<\tau_n<\dots<\tau^1_*<\frac{1}{4}$,  ËùÒÔ
\begin{equation}\label{T206}
  2\pi\Gamma\sin(2\pi\xi_{21})<2\pi\Gamma\sin(2\pi\tau^0_*)=\frac{5(1-r)}{3\pi\Gamma}.
\end{equation}
ÓÉ\ (\ref{T205}), (\ref{T206}) µÃ \ $2\pi\Gamma\sin(2\pi\xi_{21})(\xi_{20}-\tau_0)<(1-r)\Delta_0$, ŽúÈë\ (\ref{T204}) µÃ
\begin{equation}\label{T207}
  \Delta_0<\frac{W_0}{r}.
\end{equation}
 ¶ÔµÚ¶þŽÎʱŒäŒäžô \ $\Delta_1$, ͬÑù¿ÉµÃ \ $\Delta_1<\frac{W_1}{r}$, ÇÒ
 \begin{eqnarray*}
 % \nonumber to remove numbering (before each equation)
   W_1 &=& -rW_0+2r\Delta_0+r\Gamma[\cos(2\pi\tau_2)-\cos(2\pi\tau_1)] \\
      &<&  r\Big(-W_0+2\Delta_0-\frac{1-r}{r}\Delta_0\Big)\\
    &<&  W_0\Big[1-\frac{(r-1)^2}{r}\Big].
 \end{eqnarray*}
ÓÉÓÚ \ $\xi_{21}\in(\tau_0,\tau^1_*)$, ËùÒÔ \ $2\pi\Gamma\sin(2\pi\xi_{21})>2\pi\Gamma\sin(2\pi\tau_0)=\frac{1-r}{r}$, ŽÓ¶ø\ $W_1<W_0\Big[1-\frac{(r-1)^2}{r}\Big]$, ÒòŽË
\begin{equation}
   \Delta_1<\frac{W_1}{r}<\frac{W_0(1-\frac{(r-1)^2}{r})}{r}.
   \end{equation}
ÒÔŽËÀàÍÆ\ $\Delta_n<\frac{W_0[1-\frac{(r-1)^2}{r}]^n}{r}$.

ŽÓ¶øÕð²üʱŒä \ \emph{T} Ϊ
 \begin{equation}
 T=\sum_{n=0}^{\infty}\Delta_n<\sum_{n=0}^\infty\frac{W_0[1-\frac{(r-1)^2}{r}]^n}{r}=\frac{W_0}{(1-r)^2},
\end{equation}
ÓÉ\ (\ref{T201}) µÃ
\begin{equation}
  \frac{W_0}{(1-r)^2}\leq\tau^1_*-\tau_0.
\end{equation}Ö€±Ï.
\section{\kaishu Õð²üʱŒäµÄ¹ÀŒÆ}

{\heiti \textbf{¶šÀí\ $\ref{T3}$ µÄÖ€Ã÷} }
ÕâÀïÌÖÂÛϵͳ\ (\ref{tau}-\ref{W}) ÔÚŒ«ÏÞÇéÐÎ\ $\pi\Gamma\rightarrow1$ ϵÄÖÜÆÚÕð²ü, Áî\ $\epsilon=\pi\Gamma-1$. ÎÒÃǵÄÄ¿±êÊÇÔÚ\ $\epsilon\rightarrow 0$ ʱµÃµœÒ»ÁÐÊÕÁ²µÄʱŒä¶ÎÐòÁÐÀŽ¹ÀŒÆÕûžöÕð²üʱŒä\ \emph{T}. ·œ·šÊÇÓÃ\ Taylor չʜÀŽ¹ÀŒÆÿŽÎÅöײŒäžôµÄʱŒä. Ê×ÏÈ¿ŒÂÇÔÚÏžöÖÜÆÚ¿ªÊŒÊ±µÄ·ÖÀëµã. Éè\ $\tau=\tau_{0}$ Ϊ³õÊŒµÄ·ÖÀëʱ¿Ì, \ $\tau=\tau_{0}$ ÓÉ$\pi\Gamma\sin(2\pi\tau_{0})=1$ ÀŽŸö¶š, ¶ÔÓÚ\ $\epsilon\rightarrow 0$ ʱ, µ¯ÌøÇòÀ뿪Ɯ°åºó, Éè\ $\tau_{1}$ Ϊµ¯ÌøÇòÓëÆœ°åµÚÒ»ŽÎÅöײµÄʱŒä. È»ºóµ¯ÌøÇòÔÚÓÐÏÞµÄʱŒäÄÚÓëÆœ°åœøÐÐÁËÎÞÊýŽÎÅöײ, ×îºóŸ²Ö¹ÔÚÆœ°åÉÏ.

ÓÉ\ (\ref{tau}) ¿ÉµÃµ¯ÌøÇòÓëÆœ°åµÚ\ \emph{n} ŽÎ·ÖÀëºóµÄÏà¶ÔŸàÀë·œ³ÌΪ
\begin{equation}\label{8}
  X(\tau)=\frac{\Gamma}{2\pi}[\sin(2\pi\tau_{n})-\sin(2\pi\tau)]+[W_{n}+\Gamma\cos(2\pi\tau_{n})](\tau-\tau_{n})-(\tau-\tau_{n})^{2}
\end{equation}
ÉèÿŽÎÅöײµÄœâΪ
\begin{equation}\label{9}
  x_{n}(\tau)=x(\tau),\tau_{n}\leq\tau\leq\tau_{n+1},
\end{equation}
³õÊŒÌõŒþΪ\ $x_{n}(\tau_{n})=0$, ÇÒÓÐ\ $W_{n}=-rW_{n}^{(-)}$. ³õʌֵ\ $(\tau_0,W_0)$ Âú×ã\
\begin{center}
$\pi\Gamma\sin2\pi\tau_{0}=1$
\end{center}
 ºÍ\ $W_{0}=0$.
°Ñ\ (\ref{8}) µÄœâ\ $ x_n$ ÔÚ³õʌֵ\ $\tau=\tau_{n}$ ʱ, ÓÃ\ Taylor Õ¹¿ªµœ\ \emph{M} œ×µÃ:
\begin{equation}\label{10}
  x_{n}(t)=\sum_{m=o}^{M} B_{m}^{n}(\tau-\tau_{n})^{m},
\end{equation}
ÆäÖÐ
\begin{equation}\label{11}
  B_{m}^{n}\equiv\frac{1}{m!}\frac{d^{m}x_{n}}{d\tau^{m}}(\tau_{n}).
\end{equation}
\emph{M} ΪŽý¶šµÄÖµ. µÚ\ \emph{n} ŽÎÅöײʱ¿Ì\ $\tau_{n+1}=\tau_{n}+\Delta_{n}$ ÓɵÈÊœ\ $x_{n}(\tau_{n+1})=0$ Ÿö¶š, ÒòΪ\ $x_{n}(\tau_{n})=0$ ,
ŽúÈë\ (\ref{10}) ¿ÉµÃ\ $B_{0}^{n}=0$, ŒŽ
\begin{equation}\label{12}
  \sum_{m=1}^{M}B_{m}^{n}(\Delta_{n})^{m}=0,
\end{equation}
(\ref{12}) ÊœµÈŒÛÓÚ:
\begin{equation}\label{13}
  \Delta_{n}\sum_{m=1}^{M}B_{m}^{n}(\Delta_{n})^{m-1}=0,
\end{equation}
ŒŽ
\begin{equation}\label{14}
  \Delta_{n}\sum_{m=0}^{M-1}B_{m+1}^{n}(\Delta_{n})^{m}=0.
\end{equation}
ϵÊý\ $B_{m+1}^{n}$ ¿ÉŽÓ\ (\ref{8}) ÖжÔʱŒä\ $\tau$ Ç󵌵óö, ÇóµŒ\ $2m$ ŽÎµÃ³ö
\begin{equation}\label{15}
  B_{2m}^{n}=\frac{(-1)^{m+1}}{(2m)!}(2\pi)^{2m-1}\Gamma\sin(2\pi\tau_{n}),m\geq2,
\end{equation}
ÇóµŒ\ $2m+1$ ŽÎµÃ³ö
\begin{equation}\label{16}
  B_{2m+1}^{n}=\frac{(-1)^{m+1}}{(2m+1)!}(2\pi)^{2m}\Gamma\cos(2\pi\tau_{n}),m\geq1.
\end{equation}
ÔòµÃ¶ÔËٶȵÄ\ Taylor չʜΪ
\begin{equation}\label{17}
  \dot{x}_{n}(\tau)=\sum_{m=1}^{M} mB_{m}^{n}(\tau-\tau_{n})^{m-1}
  =\sum_{m=0}^{M-1} (m+1)B_{m+1}^{n}(\tau-\tau_{n})^{m}.
\end{equation}
¶ÔÓÚ³õʌֵ\ $x_0(\tau_0)=0,W_0=0$ ŽæÔÚ
\begin{equation}\label{19}
 B_{1}^{0}=W_{0}=0,
\end{equation}
\begin{equation}\label{20}
  B_{2}^{0}=-1+\pi\Gamma\sin(2\pi\tau_{0})=0.
\end{equation}
ÓÉ\ (\ref{11}) ºÍ\ (\ref{17}) ¿ÉµÃµ¯ÌøÇòÓëÆœ°åÅöײʱ
\begin{equation}\label{21}
  B_{1}^{n}=\dot{x}_{n}(\tau_{n})=-r\dot{x}_{n-1}(\tau_{n})
  =-r\sum_{m=0}^{M-1}(m+1)B_{m+1}^{n-1}(\Delta_{n-1})^{m}.
\end{equation}
Èô\ $\tau_{n}$ ºÍϵÊý\ $B_{m}^{n}$ ¿ÉÒÔÓÃ\ $\epsilon$ µÄÃÝŽÎÕ¹¿ª, ÄÇÃŽ¿ÉµÃ\ $\Delta_{n}$¹ØÓÚ\ $\epsilon$ µÄÃÝŽÎÕ¹¿ª. œâ\ $x_{0}(\tau_{0})$ ºÍ\ $x_{n}(\tau), n\geq1$ ÓÐËù²»Í¬, ÕâÊÇÒòΪ\ $\dot{x}_{0}(\tau_{0})=0$, ¶øµ±\ $n\geq1$ ʱ\ $\dot{x}_{n}(\tau_{n})\neq0$. Ëæºó»áÖ€Ã÷\ $B_{m}^{0}$ ÊÇ\ $\epsilon$ µÄÃÝ×é³ÉµÄչʜ, ÇÒÊÇ\ $\epsilon$ µÄ·ÖÊýŽÎÃÝ, ÐÎʜΪ\ $\epsilon^{1/2}$, $\epsilon^{3/2}$, $\epsilon^{5/2}$ µÈµÈ, µ±\ $n\geq1$ ʱҲͬÑù°üº¬\ $\epsilon$ µÄÕûÊýŽÎÃÝ. ÓÉÓÚ·ÖÊýŽÎÃݵijöÏÖ, ÎÒÃÇ¿ÉÒÔÔÚ\ $\epsilon\rightarrow0$ ʱ¹ÀŒÆÕûžöÕð²üʱŒä\ \emph{T}.

Ê×ÏÈ¿ŽµÚÒ»žöœâ\ $x_{0}(\tau)$, µ±\ $n=0$ ʱ¿ÉÓÉ(\ref{8})µÃ³ö. ³õÊŒ·ÖÀëʱ¿ÌÂú×ã\
\begin{center}
$\pi\Gamma\sin2\pi\tau_{0}=1$,
\end{center}
ŒŽ\ $\tau=\tau_{0}$ Âú×ãÏÂÊœ
\begin{equation}\label{22}
  \sin(2\pi\tau_{0})=\frac{1}{\pi\Gamma}=\frac{1}{1+\epsilon},
\end{equation}
ÔòÓÉÈýœÇº¯ÊýÐÔÖʵÃ:
\begin{equation}\label{23}
  \cos(2\pi\tau_{0})=\frac{\sqrt{\epsilon(2+\epsilon)}}{1+\epsilon}.
\end{equation}
ΪÁËŒÆËãµÚÒ»ŽÎÅöײʱ¿Ì\ $\tau_{1}$, ÓÉ\ (\ref{10}) È¡\ $n=0$ Œ°\ (\ref{19}) ºÍ\ (\ref{20}) ¿ÉµÃ
\begin{equation}\label{24}
  x_{0}(\tau)=\sum_{m=3}^{M}B_{m}^{0}(\tau-\tau_{0})^{m},
\end{equation}
ÓÉ\ (\ref{13}) È¡\ $n = 0$ ¿ÉµÃ
\begin{equation}\label{25}
  \sum_{m=3}^{M}B_{m}^{0}(\Delta_{0})^{m-3}=0.
\end{equation}
ÏÖÔÚœ«\ $B_{m}^{0}$ ÓÃ\ $\epsilon$ µÄÃÝŽÎÕ¹¿ª, ÔÚ\ (\ref{15}) ºÍ\ (\ref{16}) ÖÐÈ¡\ $n = 0$ ÔÙÓÉ\ (\ref{22}) ºÍ\ (\ref{23}). $\epsilon$ µÄÃÝŽÎÒÀÀµÓÚ \ $\Gamma\cos2\pi\tau_{0}$, ÓÉ\ (\ref{23}) ʹÓÃ\ Taylor չʜ¹ØÓÚ\ $\epsilon$ Õ¹¿ªµÃ
\begin{equation}\label{26}
  \pi\Gamma\cos2\pi\tau_{0}=\sqrt{2}\sqrt{\epsilon}\sqrt{1+\frac{\epsilon}{2}}
  =\sqrt{2}\sqrt{\epsilon}\Big(1+\frac{\epsilon}{4}-\frac{\epsilon^{2}}{32}+O(\epsilon^{3})\Big).
\end{equation}
ÔòϵÊý\ $B_{m}^{0}$ ¿ÉÒÔÐŽ³É:
\begin{equation}\label{27}
  B_{m}^{0}=b_{m0}^{0}+b_{m1}^{0}\epsilon^{1/2}+b_{m3}^{0}\epsilon^{3/2}+b_{m5}^{0}\epsilon^{5/2}
  +O(\epsilon^{7/2})=b_{m0}^{0}+\sum_{l=0}^{L}b_{m,2l+1}^{0}\epsilon^{(2l+1)/2}.
\end{equation}
 µ±\ $m\geq3$ ʱ, ÓÐ\ $b_{30}^{0}=0$, $B_{2m}^{0}=b_{2m,0}^{0}$, Õ¹¿ªµœ\ \emph{L} œ×Ϊֹ, \emph{L} ΪŽý¶šµÄÊýÖµ. ϵÊý\ $b_{m,2l+1}^{0}$
 ¿ÉÓÉ\ (\ref{16}) ºÍ \ (\ref{26}) Çó³ö. ÓÉ\ $B_{m}^{0}$ µÄÃݎΜṹʹµÃ\ $\Delta_{n}$ ¿ÉÕ¹¿ª³É:
\begin{equation}\label{28}
\Delta_{n}=\eta_{0}^{n}+\eta_{1}^{n}\epsilon^{1/2}+\eta_{2}^{n}\epsilon+\eta_{3}^{n}\epsilon^{3/2}
+\eta_{4}^{n}\epsilon^{2}+\eta_{5}^{n}\epsilon^{5/2}+O(\epsilon^{7/2}),
\end{equation}
µ±\ $ n = 0$ ʱΪ
\begin{equation}\label{29}
  \Delta_{0}=\eta_{0}^{0}+\eta_{1}^{0}\epsilon^{1/2}+\eta_{2}^{0}\epsilon+\eta_{3}^{0}\epsilon^{3/2}
+\eta_{4}^{0}\epsilon^{2}+\eta_{5}^{0}\epsilon^{5/2}+O(\epsilon^{7/2}).
\end{equation}
œ«\ (\ref{27}) ºÍ\ (\ref{29}) ŽúÈë\ (\ref{25}) µÃ
\begin{multline}\label{30}
  \sum_{m=3}^{M}\Big(b_{m0}^{0}+b_{m1}^{0}\epsilon^{1/2}+b_{m3}^{0}\epsilon^{3/2}+\cdots\Big)\times \\
  \Big(\eta_{0}^{0}+\eta_{1}^{0}\epsilon^{1/2}+\eta_{2}^{0}\epsilon+\eta_{3}^{0}\epsilon^{3/2}+\eta_{4}^{0}\epsilon^{2}+\cdots \Big)^{m-3}=0.
\end{multline}
¿ÉÒԵõœ¹ØÓÚϵÊý\ $\eta_{l}^{0}$ µÄµÝÍÆ·œ³Ì. Ê×ÏÈ×¢Òⵜ\ $b_{30}^{0}=0$, ÒòŽË\ $\epsilon^{0}$ µÄϵÊý\ $\eta^0_0$ Âú×ã:
\begin{equation}\label{31}
  b_{40}^{0}\eta_{0}^{0}+b_{50}^{0}(\eta_{0}^{0})^{2}+\dots=\sum_{m=4}^{M}b_{m0}^{0}(\eta_{0}^{0})^{m-3}=0,
\end{equation}
(\ref{31})ÊÇ\ $\eta_{0}^{0}$ µÄ\ $M-3$ ŽÎÆëŽÎ·œ³Ì, ×ÜÓÐœâ\ $\eta_{0}^{0}=0$, ÒòΪ\ $\eta_{0}^{0}=0$ ÊÇΚһÂú×ãÕð²üÔÚ\ $\epsilon\rightarrow0$ ʱ, Õð²üʱŒä\ \emph{T} Ç÷ÓÚ\ 0 µÄœâ. ÔÚ\ (\ref{30}) °Ž\ $\epsilon^{1/2}$ µÄÉýÃÝÅÅÁÐ, ÓÉ\ $\eta_{0}^{0}=0$, $b_{30}^{0}=0$ ºÍ\ $B_{2m,0}^{0}=b_{2m,0}^{0}$ ¿ÉµÃ
\begin{multline}\label{32}
 [b_{40}^{0}\eta_{1}^{0}+b_{31}^{0}]\epsilon^{1/2}+[b_{40}^{0}\eta_{2}^{0}+b_{50}^{0}(\eta_1^0)^2]\epsilon \\
 +[b_{40}^{0}\eta_{3}^{0}+2b_{50}^{0}\eta_1^0\eta_2^0+ b_{60}^{0}(\eta_{1}^{0})^{3}+ b_{51}^{0}(\eta_{1}^{0})^{2}+b_{33}^{0}]\epsilon^{3/2}+O(\epsilon^{2})=0.
\end{multline}
¿ÉµÃ¹ØÓÚϵÊý\ $\eta_{l}^{0}$ µÄ·œ³Ì:
\begin{eqnarray}
% \nonumber to remove numbering (before each equation)
  b_{40}^{0}\eta_{1}^{0}+b_{31}^{0} &=& 0,\label{33} \\
  b_{40}^{0}\eta_{2}^{0}+b_{50}^{0}(\eta_1^0)^2 &=& 0, \label{34}\\
  b_{40}^{0}\eta_{3}^{0}+2b_{50}^{0}\eta_1^0\eta_2^0+ b_{60}^{0}(\eta_{1}^{0})^{3}+ b_{51}^{0}(\eta_{1}^{0})^{2}+b_{33}^{0} &=& 0.\label{35}
\end{eqnarray}
 \ $\eta_{1}^{0}$ ŽæÔÚµÄÌõŒþ, Ö»Ðè\ (\ref{24}) ÖÐÁî\ $M = 4$. ÓÉ\ (\ref{33}), (\ref{34}) ºÍ\ (\ref{35}) µÃ
 \begin{eqnarray}
 % \nonumber to remove numbering (before each equation)
  \eta_{1}^{0} &=& -\frac{b_{31}^{0}}{b_{40}^{0}},\label{36} \\
   \eta_{2}^{0} &=& -\frac{b_{50}^{0}(\eta_1^0)^2}{b_{40}^{0}}, \\
   \eta_{3}^{0} &=& -\frac{2b_{50}^{0}\eta_1^0\eta_2^0+b_{60}^{0}(\eta_{1}^{0})^{3}+ b_{51}^{0}(\eta_{1}^{0})^{2}+b_{33}^{0}}{b_{40}^{0}}.
 \end{eqnarray}
$$\cdots$$

ΪÁËœÓ×ÅŒÆËã\ $\tau_{n}$, $n\geq2$. Ê×ÏÈÐèÇó³öµÚÒ»ŽÎÅöײʱµÄËÙ¶È\ $\dot{x}_{0}(\tau_{1})$. ÓÉ\ (\ref{17}) ÔÚ\ $n = 0$ ʱµÃ:
\begin{equation}\label{39}
  \dot{x}_{0}(\tau)=\sum_{m=0}^{M-1} (m+1)B_{m+1}^{0}(\tau-\tau_{0})^{m},
\end{equation}
ÁªÁ¢\ $B_{1}^{0}=B_{2}^{0}=0$ ºÍչʜ\ (\ref{27}) ¿ÉµÃ:
\begin{equation}\label{40}
  \dot{x}_{0}(\tau_{1})=\nu_{3}^{0}\epsilon^{3/2}+\nu_{4}^{0}\epsilon^{2}+O(\epsilon^{5/2}),
\end{equation}
ÆäÖÐ
\begin{equation}\label{41}
\nu_{3}^{0}=\frac{16\sqrt{2}}{\pi}.
\end{equation}
²¢ÇҿɵÃ
  \begin{equation}\label{43}
    B_{1}^{1}=\dot{x}_{1}(\tau_{1})=-r\dot{x}_{0}(\tau_{1})=b_{13}^{1}\epsilon^{3/2}+O(\epsilon^{5/2}),
  \end{equation}
ÆäÖÐ\ $b_{13}^{1}=-r\nu_{3}^{0}$.
ÏÂÃæÎÒÃÇÀŽÖ€Ã÷ϵÊý\ $B_{1}^{n}$ Âú×ãÏÂÊœ:
\begin{equation}\label{45}
 B_{1}^{n}=b_{13}^{n}\epsilon^{3/2}+b_{14}^{n}\epsilon^{2}+O(\epsilon^{5/2}),
\end{equation}
ÔÚ\ (\ref{15}) ºÍ\ (\ref{16}) ÖÐ\ $\pi\Gamma\sin2\pi\tau_{n}$ ºÍ\ $\pi\Gamma\cos2\pi\tau_{n}$ ¿ÉÕ¹³É\ $\epsilon$ µÄÃÝŽÎÕ¹¿ª, ŒŽ

\begin{eqnarray}
% \nonumber to remove numbering (before each equation)
  \pi\Gamma\sin2\pi\tau_{n} &=& \pi\Gamma\sin2\pi\Big(\tau_{0}+\sum_{k=0}^{n-1}\Delta_{k}\Big), \\
  \pi\Gamma\sin2\pi\tau_{n} &=& \pi\Gamma\sin2\pi\Big[\tau_{0}+\sum_{l=1}^{L}\Big(\sum_{k=0}^{n-1}\eta_{k}\Big)\epsilon^{l/2}\Big], \\
  \pi\Gamma\sin2\pi\tau_{n} &=& \pi\Gamma\sin2\pi\Big(\tau_{0}+\sum_{l=1}^{L}\mu_{l}^{n}\epsilon^{l/2}\Big),\label{46}
\end{eqnarray}
ÆäÖÐ
\begin{equation}\label{47}
  \mu_{l}^{n}=\sum_{k=0}^{n-1}\eta_{l}^{k},
\end{equation}
$\eta_{0}^{n}=0$, ÒòΪ\ $\epsilon\rightarrow0$, ËùÒÔÕð²üʱŒäÇ÷ÓÚ\ 0, ËùÒÔ\ $\mu_{0}^{n}=0$.
ͬÀíÓÐ
\begin{equation}\label{48}
  \pi\Gamma\cos2\pi\tau_{n}=\pi\Gamma\cos2\pi\Big(\tau_{0}+\sum_{l=1}^{L}\mu_{l}^{n}\epsilon^{l/2}\Big).
\end{equation}
Ôò\ (\ref{46}) ºÍ\ (\ref{48}) ¿ÉœøÒ»²œÐŽ³ÉÏÂÁÐÐÎÊœ:
\begin{eqnarray*}
% \nonumber to remove numbering (before each equation)
 \pi\Gamma\sin2\pi\tau_{n} &=& \sum_{l=0}^{L}\phi_{l}^{n}\epsilon^{l/2},\\
  \pi\Gamma\cos2\pi\tau_{n} &=& \sum_{l=0}^{L}\psi_{2l+1}^{n}\epsilon^{(2l+1)/2},
\end{eqnarray*}
ÆäÖÐ\ $\phi_{0}^{n}=1$, $\phi_{l}^{n}=0$.
¶ÔÓÚϵÊý\ $B_2^n$ ÓÐ
\begin{equation}\label{51}
  B_{2}^{n}=b_{22}^{n}\epsilon+O(\epsilon^{3/2}),
\end{equation}
ÆäÖÐ\ $b_{22}^{n}=\phi_{2}^{n}$, ¶ÔÓÚϵÊý\ $B_3^n$ ÓÐ
\begin{equation}\label{52}
  B_{3}^{n}=b_{31}^{n}\epsilon^{1/2}+O(\epsilon),
\end{equation}
ÆäÖÐ\ $ b_{31}^{n}=\frac{2}{3}\pi\psi_{1}^{n}$, ¶ÔÓÚϵÊý\ $B_4^n$ ÓÐ
\begin{equation}\label{53}
  B_{4}^{n}=b_{40}^{n}+O(\epsilon^{1/2}).
\end{equation}
ÆäÖÐ\ $b_{40}^{n}=-\frac{\pi^{2}}{3}$.
ÔÚ\ (\ref{21}) ÖÐ,\ $B_{1}^{n-1}$ ÓÐÏàËƵĜṹ, ¿ÉµÃ
\begin{eqnarray*}
% \nonumber to remove numbering (before each equation)
   B_{1}^{n} &=&-r\dot{x}_{n-1}(\tau_n)=-rB_1^{n-1}-2rB_2^{n-1}\Delta_{n-1}-3rB_3^{n-1}\Delta_{n-1}^2\cdots\\
    &=& -rB_1^{n-1}-2rB_2^{n-1}(\eta_{1}^{n-1}\epsilon^{1/2}+\cdots)-3rB_3^{n-1}(\eta_{1}^{n-1}\epsilon^{1/2}+\cdots)^2\cdots\\
    &=& -r\Big[b_{13}^{n-1}+2b_{22}^{n-1}\eta_{1}^{n-1}+3b_{31}^{n-1}(\eta_{1}^{n-1})^{2}+4b_{40}^{n-1}(\eta_{1}^{n-1})^{3}\Big]\epsilon^{3/2}+\cdots\\
    &=& b_{13}^{n}\epsilon^{3/2}+\cdots.\label{57}
\end{eqnarray*}

ÓÉ\ $B_{1}^{n+1}=\dot{x}_{n+1}(\tau_{n+1})=W_{n+1}$, ¿ÉµÃÿŽÎÅöײʱµÄËٶȿÉÒÔÕ¹³É:
\begin{equation}\label{60}
  \dot{x}_{n}(\tau_{n+1})=\nu_{3}^{n}\epsilon^{3/2}+\cdots,
\end{equation}
ÆäÖÐ
\begin{equation}\label{61}
  \nu_{3}^{n}=b_{13}^{n}+2b_{22}^{n}\eta_{1}^{n}+3b_{31}^{n}(\eta_{1}^{n})^{2}+4b_{40}^{n}(\eta_{1}^{n})^{3},
\end{equation}
²¢ÇÒÓÐ
\begin{equation}\label{62}
  b_{13}^{n}=-r\nu_{3}^{n-1}.
\end{equation}
œ«\ (\ref{51}), (\ref{52}), (\ref{53}) ºÍ\ (\ref{28}) ŽúÈë\ (\ref{12}) µÃ³öÒ»žö¹ØÓÚ\ $\eta_{1}^{n}$ µÄÈýŽÎ·œ³Ì:
\begin{equation}\label{63}
   b_{40}^{n}(\eta_{1}^{n})^{3}+b_{31}^{n}(\eta_{1}^{n})^{2}+b_{22}^{n}\eta_{1}^{n}+b_{13}^{n}=0.
\end{equation}
ÔÚ\ (\ref{46}) ºÍ\ (\ref{48}) µÄϵÊýÖÐÓÐ
\begin{eqnarray*}
% \nonumber to remove numbering (before each equation)
  \phi_{0}^{n} &=& 1 ,\\
  \phi_{2}^{n} &=& 2\pi\sqrt{2}\mu_{1}^{n}-2\pi^{2}(\mu_{1}^{n})^{2} ,\\
  \psi_{1}^{n} &=& \sqrt{2}-2\pi\mu_{1}^{n} .
\end{eqnarray*}

ÓÉ\ (\ref{47}), (\ref{61}), (\ref{62}) ºÍ\ (\ref{63}) ×é³ÉµÄ·œ³Ì×éÐγÉÒ»žö¹ØÓÚϵÊý\ $\eta_{l}^{n}$ µÄµÝÍƹ«Êœ, ÓɎ˿ɵóöÕûžöÕð²üʱŒäµÄ¹ÀŒÆ. ·œ³Ì\ (\ref{36}) ºÍ \ (\ref{41}) ÌṩµÝÍƹ«ÊœµÄ³õʌֵ. ÒýÈë²ÎÊý
\begin{center}
 $w_{n}=\pi\nu_{3}^{n}$,\ \ $y_{n}=\pi \eta_{1}^{n}$,\ \ $z_{n}=\pi\mu_{1}^{n}$
\end{center}
ÀŽ¹æ·¶Ëã·šÐÎÊœ. ÔòÒýÈëºóµÃ³õʌֵΪ
\begin{eqnarray}
  w_{0} &=& 16\sqrt{2}, \\
  y_{0} &=& 2\sqrt{2}, \\
  z_{n} &=& \sum_{k=0}^{n-1}y_{k},\ n\geq1.
 \end{eqnarray}
 $y_n$ ºÍ\ $w_n$ Âú×ãÈçÏ·œ³Ì:
\begin{equation}\label{70}
  y_{n}^{3}+(4z_{n}-2\sqrt{2})y_{n}^{2}+(6z_{n}^{2}-6\sqrt{2}z_{n})y_{n}+r3w_{n-1}=0,\ n\geq1,
\end{equation}
\begin{equation}\label{71}
  w_{n}= -rw_{n-1}+4\Big(\sqrt{2}z_{n}-z_{n}^{2}\Big)y_{n}+2\Big(\sqrt{2}-2z_{n}\Big)y_{n}^{2}-\frac{4y_{n}^{3}}{3},\ n\geq1.
\end{equation}
·œ³Ì\ (\ref{70}) ÊǹØÓÚ\ $y_{n}$ µÄÈýŽÎ·œ³Ì; ¿ÉÇó³öÆäÖÐÖ»ÓÐÒ»žöʵžù.
ÀûÓÃ\begin{equation}\label{72}
  y_n=z_{n+1}-z_{n},
\end{equation}
ÔÙÓÉ\ (\ref{70}) ºÍ\ (\ref{71}) ¿ÉµÃ
\begin{multline}\label{73}
  (z_{n+2}-z_{n+1})^{3}+(4z_{n+1}-2\sqrt{2})(z_{n+2}-z_{n+1})^{2}+(6z_{n+1}^{2}-6\sqrt{2}z_{n+1})(z_{n+2}-z_{n+1})+ \\
  r[-3(z_{n+1}-z_{n})^3+(4\sqrt{2}-8z_n)(z_{n+1}-z_{n})^2+(6\sqrt{2}z_n-6z_n^2)(z_{n+1}-z_{n})]=0
\end{multline}
ΪÁ˱£Ö€Ëã·šµÄÒ»ÖÂÐÔ, ÒòΪ¶ÔÓڹ̶šµÄ\ $\epsilon$, ÍêÕûµÄÕð²üʱŒä\ \emph{T} ÊÇÓÐÏÞµÄ. ÊýÁÐ\ $(y_{n})_{n=0}^{\infty}$ ÐèÊÕÁ²ÓÚ\ 0, ¶øÇÒÏÂÊöŒ«ÏÞŽæÔÚ
\begin{equation}\label{74}
  \lim_{n\rightarrow\infty}z_{n}=\sum_{n=0}^{\infty}y_{n}.
\end{equation}
²¢Éè
\begin{equation}\label{75}
  \lambda(r)=\lim_{n\rightarrow\infty}z_{n}=\sum_{n=0}^{\infty}y_{n}.
\end{equation}
¶ÔÓÚÈ·¶šµÄ\ \emph{r}, ³õʌֵ\ $z_1=y_0=2\sqrt{2}$, $z_2$ ¿ÉÓÉ\ (\ref{70}) Çó³ö. Ôò×îÖÕ\ $\lambda$ ¿ÉÓÉ\ (\ref{73}) µüŽúÇó³ö. ×îºó, Õð²üʱŒäÔÚ\ $\pi\Gamma\rightarrow 1$ ʱΪ
\begin{equation}\label{f1}
  T=\sum_{n=0}^{\infty}\Delta_{n}\sim\sum_{n=0}^{\infty}\eta_{n}\sqrt{\epsilon}=\frac{1}{\pi}\sum_{n=0}^{\infty}y_{n}\sqrt{\epsilon}=\frac{1}{\pi}\lambda\sqrt{\epsilon},
\end{equation}
ÕâµÈŒÛÓÚÏÂÊœ£º
\begin{equation}\label{Final}
  T\cong\frac{\lambda(r)}{\pi}\sqrt{\pi\Gamma-1}.
\end{equation}
Ö€±Ï.

ÔÚ¶šÀíÖ€Ã÷ÖгöÏֵĺ¯Êý\ $\lambda(r)$ µÄœâÎö±íŽïÊœÎÞ·šÇó³öÀŽ, µ«¶Ôžø¶šµÄ\ $r\in(0,1)$, ÎÒÃÇ¿ÉÇó³öÆäœüËÆÖµ, ÓÉŽË»­³öÆ亯ÊýÍŒÏó, ŒûÍŒ\ 2.



\chapter{\heiti ÊýֵģÄâ}

ÕâÒ»ÕÂÖÐ, ÎÒÃÇœ«žø³öµ¯ÌøÇòÄ£ÐÍ\ (\ref{tau}-\ref{W}) ÔÚÂú×㶚Àí\ $\ref{T1}$ ºÍ\ $\ref{T2}$ µÄÌõŒþʱµÄÊýÖµŒÆËãœá¹û, ÒÔŽËÀŽÑéÖ€µŒ³öµÄÀíÂÛœá¹û. Õð²üÊÇ΢СÕñ·ùµÄÕñ¶¯, ÍêÕûÕð²üÔòÊÇÔÚÓÐÏÞµÄʱŒäÀï°üÀšÁËÎÞÊýŽÎµÄÕñ¶¯. ÔÚÒÔÏÂÊýֵģÄâÍŒÖÐ, ×Ý×ø±ê\ $X(\tau)$ Žú±íÎÞÁ¿žÙ»¯ÒÔºóµ¯ÌøÇòÄ£ÐÍ\ (\ref{tau}-\ref{W}) Öе¯ÌøÇòÓëÆœ°åÖ®ŒäµÄÏà¶ÔŸàÀë, ºá×ø±ê\ $\tau$ Žú±íÎÞÁ¿žÙÒÔºóµ¯ÌøÇòÄ£ÐÍ\ (\ref{tau}-\ref{W}) µÄʱŒä, µ¯ÌøÇòµÄλÖÃÓÉÍŒÖеÄÔ²»·±íÊŸ. ÔÚËæºóµÄÊýֵģÄâÍŒÖгöÏÖµÄÃÜŒ¯µÄÔ²»·ÖصþÏÖÏóÊÇÓÉÓÚµ¯ÌøÇò¶ÌʱŒäÄÚÓëÆœ°åŒä·¢ÉúÁËÎÞÇîŽÎµÄÅöײµŒÖµÄ.

¶šÀí\ $\ref{T1}$ µÄÊýֵģÄâ, ÎÒÃÇÑ¡È¡²»Í¬µÄϵͳ²ÎÊýÓë³õʌֵÀŽÑéÖ€:

(1)ÎÒÃÇѡȡϵͳµÄ²ÎÊýΪ\ $r=0.05$, $\Gamma=\frac{10}{\pi}$,
³õʌʱŒäµ±\ $\tau_0=0$ ʱ, ÓɶšÀí\ $\ref{T1}$ ¿ÉµÃ\ $W_0$ ÐèÂú×ã\ $W_0\leq \frac{1}{7}(1-13r)(\tau^1_*-\tau_0)$, ¿ÉÈ¡\ $W_0=0.00024$. Ä£Äâœá¹ûŒûÍŒ\ 3:


ÍŒ\ 3 ÖгöÏÖÔÚ\ $\tau=0.00025$ žœœüµÄÃÜŒ¯µÄÔ²»·ÖصþÏÖÏó, Æä·ÅŽó͌Ϊ͌\ 4:

ÔÚÍŒ\ 4 Öл¹»á³öÏÖÃÜŒ¯µÄÔ²»·Öصþ, œ«Æä·ÅŽóºó»¹ÊǺÍÍŒ\ 4 œá¹¹ÏàËƵÄÍŒÐÎ, ÆäÔ­ÒòÊǵ¯ÌøÇòÔÚÖØžŽÐ¡Õñ·ùÕñ¶¯.

(2)ϵͳ²ÎÊý\ $r=0.076$, $\Gamma$ ÈÔΪ\ $\frac{10}{\pi}$ ʱ, µ±³õʌʱŒä\ $\tau_0=0$ ʱ, ÓɶšÀí\ $\ref{T1}$ ¿ÉµÃ\ $W_0$ ÐèÂú×ã\ $W_0\leq \frac{1}{7}(1-13r)(\tau^1_*-\tau_0)$, ¿ÉÈ¡\ $W_0=0.000019$, ÆäÊýֵģÄâœá¹ûŒûÍŒ\ 5.

ÍŒ\ 5 ÖгöÏÖÔÚ\ $\tau=0.0000205$ žœœüµÄÃÜŒ¯µÄÔ²»·ÖصþÏÖÏó, Æä·ÅŽó͌Ϊ͌\ 6.

ÍŒ\ 3¡ª6 ±íÃ÷µ¯ÌøÇòÄ£ÐÍ\ (\ref{tau}-\ref{W}) µÄÊýֵģÄâ±íÃ÷µ±ÏµÍ³²ÎÊýÓë³õʌֵÂú×㶚Àí\ $\ref{T1}$ ʱ, µ¯ÌøÇòµÄÔ˶¯¹ìŒ£ÓëÍêÕûÕð²üµÄÃèÊöÎǺÏ.

¶ÔÓÚ¶šÀí\ $\ref{T2}$ µÄÊýֵģÄâ, ÎÒÃÇÑ¡È¡²»Í¬µÄϵͳ²ÎÊýÓë³õʌֵÀŽÑéÖ€:

(3)ÎÒÃÇѡȡϵͳ²ÎÊýΪ\ $r=0.8$, $\Gamma=\frac{10}{\pi}$, µ±³õʌʱŒä\ $\tau_0=0$ ʱ, ÓɶšÀí\ $\ref{T2}$ ÖÐ\ $W_0\leq (1-r)^2(\tau^1_*-\tau_0)$, ¿ÉµÃ\ $W_0=0.0001)$ Âú×ãÌõŒþ, ÆäÊýֵģÄâœá¹ûŒûÍŒ\ 7:

ÍŒ\ 7 ÖÐÔÚ\ $\tau=0.0005$ žœœüµÄŸÖ²¿·ÅŽóÍŒŒûÍŒ\ 8, ¿ÉŒû·ÅŽóºóµ¯ÌøÇòÔÚÖØžŽÏàËƵÄСÕñ·ùÕð²ü.

(4) ϵͳ²ÎÊýΪ\ $r=0.8$, $\Gamma=\frac{10}{\pi}$, ³õʌֵ\ $(\tau_0,W_0)$ È¡\ $(0.001,0.00006)$ ʱ, Âú×㶚Àí\ $\ref{T2}$ , ÆäÊýֵģÄâœá¹ûŒûÍŒ\ 9.
ÍŒ\ 9 ÖÐÔ²»·ÖصþÐγɵĎÖÏߟֲ¿·ÅŽóÍŒŒûÍŒ\ 10.

ÓÉÍŒ\ 10¿ÉŒûÔ²»·Öصþ²¿·ÖÊǵ¯ÌøÇòÔÚÖØžŽÏàËƵÄСÕñ·ùÕð²ü.

(5)ϵͳ²ÎÊý\ $r=0.99$, $\Gamma=\frac{10}{\pi}$ ʱ, ³õʌֵ\ $(\tau_0,W_0)$ È¡\ $(0,0.00000001)$ ʱ, Âú×㶚Àí\ $\ref{T2}$ , ÆäÊýֵģÄâœá¹ûŒûÍŒ\  11.

(6)ϵͳ²ÎÊý\ $r=0.99$, $\Gamma=\frac{10}{\pi}$ ʱ, ³õʌֵ\ $(\tau_0,W_0)$ È¡\ $(0.0001,0.000000003)$ ʱ, Âú×㶚Àí\ $\ref{T2}$ , ÆäÊýֵģÄâœá¹ûŒûÍŒ\  12.

ÍŒ\ 7¡ª12 ±íÃ÷µ¯ÌøÇòÄ£ÐÍ\ (\ref{tau}-\ref{W}) µÄÊýֵģÄ⵱ϵͳ²ÎÊýÓë³õʌֵÂú×㶚Àí\ $\ref{T2}$ ʱ, µ¯ÌøÇòµÄÔ˶¯¹ìŒ£ÓëÍêÕûÕð²üµÄÃèÊöÎǺÏ.


%%%%%%%%%%%%%%%%%%%%%%%%%%%%%%%%%%%%%%%%%%%%%%%%%%%%%%%%%%%%%%%%%%%%%%%%%
\chapter{\heiti œáÊøÓï}\label{chapterend}

%\thispagestyle{empty}
±ŸÎÄÌÖÂÛÁËÎÞ×èÄᵯÌøÇòÄ£ÐÍ\ (\ref{tau}-\ref{W}) µÄÕð²üÏÖÏó, ÀûÓÃÖÐÖµ¶šÀí, ÎÒÃÇÕÒ³öÁËÎÞ×èÄᵯÌøÇòÄ£ÐÍ\ (\ref{tau}-\ref{W}) ÔÚ\ \emph{r} ºÍ\ $1-r$ ³ä·ÖСµÄÁœÖÖÇé¿öÏ·¢ÉúÍêÕûÕð²üµÄ³ä·ÖÌõŒþ, ²¢¶Ô¶šÀí\ $\ref{T1}$ ºÍ\ $\ref{T2}$ µÄœáÂÛÓÃÊýֵģÄâœøÐÐÁËÑéÖ€. ¹ØÓÚÎÞ×èÄᵯÌøÇòÄ£ÐÍ\ (\ref{tau}-\ref{W}) ÔÚ\ $\pi\Gamma\rightarrow1$ ʱµÄÖÜÆÚÐÔÕð²üʱŒä, ÎÒÃÇžø³öÏàÓŠµÄœüËƹÀŒÆÊœ.


%%%%%%%%%%%%%%%%%%%%%%%%%%%%%%%%%%%%%%%%%%%%%%%%
{\footnotesize
\begin{thebibliography}{999}

\bibitem{Afsharnezhad}Z. Afsharnezhad and M. K. Amaleh, Continuation of the periodic orbits for the differential equation with discontinuous right hand side, {\it  Journal of Dynamics and Differential Equations} {\bf 23}(2011), 71--92.

\bibitem{Akhmet}M. U. Akhmet and D. Aru\v{g}aslan, Bifurcation of a non-smooth planar limit cycle from a vertex, {\it Nonlinear Anal. Ser. A} {\bf 71}(2009), 2723--2733.

\bibitem{Avramov} K.V. Avramov and J. Awrejcewicz, Frictional oscillations under the action of almost periodic excitation,
{\it Meccanica} {\bf 41}(2006), 119--142.

\bibitem{Awrejcewicz2} J. Awrejcewicz, M. Fe\v{c}kan and P. Olejnik,
       Bifurcations of planar sliding homoclinics, {\it Mathematical Problems in Engineering} {\bf 2006}(2006), 1--13.

\bibitem{Barroso} J. J. Barroso, M. V. Carneiro and E. E. N. Macau, Bouncing ball problem: stability of the periodic modes, {\it Physical Review E}
{\bf 79}(2011), 026206.
\bibitem{BartoliniF}G. Bartolini, A. Ferrara and E. Usani, Chattering avoidance by second-order sliding mode control, {\it Automatic control} {\bf 43}(1998), 241--246.


\bibitem{BartoliniP}G. Bartolini and E. Punta, Chattering elimination with second-order sliding modes robust to coulomb friction, {\it Journal of Dynamic Systems Measurement and Control} {\bf 122}(2000, 679--686.


\bibitem{Battelli1} F. Battelli and M. Fe\v{c}kan,
Homoclinic trajectories in discontinuous systems,
{\it J. Dynam. Differential Equations} {\bf 20}(2008), 337--376.

\bibitem{Battelli2} F. Battelli and M. Fe\v{c}kan,
Bifurcation and chaos near sliding homoclinics,
{\it J. Differential Equations}  {\bf 248}(2010), 2227--2262.

\bibitem{Battelli3} F. Battelli and M. Fe\v{c}kan,
An example of chaotic behaviour in presence of a sliding homoclinic orbit,
{\it Ann. Mat. Pura Appl.} {\bf 189}(2010), 615--642.

\bibitem{Battelli4} F. Battelli and M. Fe\v{c}kan,
Nonsmooth homoclinic orbits, Melnikov functions and chaos in discontinuous systems,
{\it Physica D} {\bf 241}(2012), 1962--1975.

\bibitem{Battelli6} F. Battelli and M. Fe\v{c}kan, Chaos in forced impact systems,
{\it Discrete and Continuous Dynamical Systems Series S } {\bf 6}(2013), 861--890.



\bibitem{Brandon}Q. Brandon, T. Ueta, D. Fournier-Prunaret and T. Kouska,
Numerical bifurcation analysis framework for autonomous piecewise-smooth dynamical systems, {\it Chaos, Solitons and Fractals} {\bf 42}(2009), 187--201.

\bibitem{BuddL} C. J. Budd and A. G. Lee, Double impact orbits of periodically forced impact oscillators, {\it Proc. R. Soc. Lond. A} {\bf 452}(1996), 2719--3750.


\bibitem{BuddD} C. Budd and F. Dux, Chattering and related behaviour in impact oscillators,{\ Philosophical Transactions of the Royal Society of London A} {\bf 347}(1994), 365--389.

\bibitem{Carmona} V. Carmona, S. Fernandez-Garcia, E. Freire and F. Torres, Melnikov theory
for a class of planar hybrid systems, {\it Physica D} {\bf 248}(2013), 44--54.

\bibitem{Casas} F. Casas, W. Chin, C. Grebogi and E. Ott, Universal grazing bifucations in impact oscillators, {\it Phys. Rev. E} {\bf 53}(1996), 134--139.

\bibitem{Chen}X. Chen and Z. Du, Limit cycles bifurcate from centers of discontinuous quadratic systems, {\it Comput. and Math. Appl.} {\bf 29}(2010), 3836--3848.

\bibitem{Chow} S.-N. Chow and S. W. Shaw, Bifurcations of subharmonics,
 {\it J. Differential Equations} {\bf 65}(1986), 304--320.

\bibitem{Coll} B. Coll, A. Gasull and R. Prohens, Degenerate Hopf bifurcations in discontinuous planar systems, {\it J. Math. Anal. Appl.} {\bf 253}(2001), 671--690.

\bibitem{Dankowicz2}
H. Dankowicz and A. B. Nordmark, On the origin and bifurcations of stick-slip oscillators,
{\it Physica D} {\bf 136}(2000), 280--302.

\bibitem{DemeioL} L. Demeio and S. Lenci, Asymptotic analysis of chattering oscillations
for an impacting inverted pendulum, {\it Quarterly Journal of Mechanics and Applied Mathematics}  {\bf 59}(2006),  419--434.

\bibitem{Drossel} B. Drossel and T. Prellberg, Dynamics of a single particle in a horizontally shaken box, {\it The European Physical Journal B-Condensed Matter and Complex Systems}, {\bf 1} (1998), 533--543.

\bibitem{DuLi} Z. Du and Y. Li, Bifurcation of periodic orbits with multiple crossings in a class of planar Filippov systems, {\it Mathematical and Computer Modelling} {\bf 55}(2012), 1072--1082.

\bibitem{DuLiShen} Z. Du, Y. Li, J. Shen and W. Zhang, Impact oscillators with homoclinic orbit
tangent to the wall, {\it Physica D} {\bf 245}(2013), 19--33.

\bibitem{Dulizhang0}Z. Du, Y. Li and W. Zhang, Bifurcation of periodic orbits in a class of planar Filippov systems, {\it Nonlinear Anal. Ser. A} {\bf 69}(2008), 3610--3628.

\bibitem{DuZhang} Z. Du and W. Zhang, Melnikov method for homoclinic
bifurcation in nonlinear impact oscillators,  {\it Comput. Math. Appl.} {\bf 50} (2005), 445--458.

\bibitem{Feckan1} M. Fe\v{c}kan, {\it Bifurcation and chaos in discontinuous and continuous systems},
Higher Education Press, Beijing, 2011.

\bibitem{Feckan2} M. Fe\v{c}kan and M. Pospšª\v{s}il, On the bifurcation of periodic orbits in discontinuous systems, {\it Commun. Math. Anal.} {\bf 8}(2 010), 87--108.

\bibitem{Feckan3} M. Fe\v{c}kan and M. Pospšª\v{s}il, Bifurcation from family of periodic orbits in discontinuous autonomous systems, {\it Differential Equations and Dynamical Systems} {\bf 20}(2012), 203--234.

\bibitem{Feckan4} M. Fe\v{c}kan and M. Pospšª\v{s}il, Bifurcation of sliding periodic orbits in periodically forced discontinuous systems£¬ {\it Nonlinear Analysis: Real World Applications} {\bf 14}(2013), 150--162.

\bibitem{Fermi} E. Fermi, On the Origin of the Cosmic Radiation, {\it Physical Review} {\bf 75} (1949), 1169.

\bibitem{Foale} S. Foale and S. R. Bishop, Dynamical complexities of forced impacting systems,
{\it Phil. Trans. R. Soc. London Ser. A} {\bf 338}(1992), 547--556.

\bibitem{Gao} J. Gao and Z. Du, Homoclinic bifurcation in a quasiperiodically excited impact inverted pendulum,
{\it Nonlinear Dynamics} (2014), 1--14.(ÒÑœÓÊÜ, Žý·¢±í)

\bibitem{Gasull}A. Gasull and J. Torregrosa, Center-focus problem for discontinuous planar differential equations, {\it Int. J. Bifurcation and Chaos} {\bf 13}(2003), 1755--1765.

\bibitem{Granados}A. Granados, S. J. Hogan and T. M. Seara,
The Melnikov method and subharmonic orbits in a piecewise-smooth system,
{\it SIAM J. Applied Dynamical Systems} {\bf 11}(2012), 801--830.

\bibitem{Gruendler} J. Gruendler,
Homoclinic solutions for autonomous ordinary differential equations with nonautonomous perturbations,
{\it Journal of Differential Equations} {\bf 122}(1995), 1--26.

\bibitem{Guck}
J. Guckenheimer and P. Holmes, {\it Nonlinear Oscillations, Dynamical Systems and
Bifurcations of Vector Fields}, Springer-Verlag, New York, 1983.

\bibitem{Han}M. Han and W. Zhang, On Hopf bifurcation in non-smooth planar systems, {\it  Journal of Differential Equations} {\bf 248}(2010), 2399--2416.

\bibitem{Hendricks}
F. Hendricks, Bounce and chaotic motion in impact print hammers, {\it IBM J} {\bf 27}(1983), 273--280.

\bibitem{Hogan1}
S. J. Hogan, On the dynamics of rigid-block motion under harmonic forcing, {\it Proc. R. Soc. London Ser. A} {\bf 425}(1989), 441--476.

\bibitem{Holmes}
P. Holmes, R. J. Full, D. Koditschek and J. Guckenheimer, Dynamics of legged locomotion: models, analysis and challengess, {\it SIAM Review} {\bf 48}(2006), 207--304.

\bibitem{Hu} N. Hu and Z. Du, Bifurcation of periodic orbits emanated from a vertex in discontinuous planar systems, {\it Communications in Nonlinear Science and Numerical Simulation} {\bf 18}(12), 3436--3448.

\bibitem{Huan}S. Huan and X. Yang, On the number of limit cycles in general planar piecewise linear systems, {\it Discrete and Continuous Dynamical Systems} {\bf 32}(2012), 2147--2164.

\bibitem{Wiggins1} K. Ide and S. Wiggins, The bifurcation to homoclinic tori in the quasiperiodically forced
Duffing oscillator, {\it Physica D} {\bf 34}(1989), 169--182.

\bibitem{Ing} J. Ing, E. Pavlovskaia, M. Wiercigroch and S. Banerjee, Bifurcation analysis of an impact
oscillator with a one-sided elastic constraint near grazing, {\it Physica D} {\bf 239}(2010), 312--321.

\bibitem{Kryzhevich} S. Kryzhevich and M. Wiercigroch, Topology of vibro-impact systems in the neighborhood of grazing,
{\it Physica D} {\bf 241}(2012), 1919--1931.

\bibitem{Leine}  R. I. Leine and H. Nijmeijer, Dynamics and bifurcations of non-smooth mechanical systems, in {\it Lecture Notes in Applied and Computational Mechanics} {\bf vol. 18}, Springer-Verlag, Berlin, 2004.

\bibitem{Levant}A. Levant, Chattering analysis, {\it Automatic Control, IEEE Transactions on} {\bf 55} (2010), 1380--1389.

\bibitem{Liang1}F. Liang and M. Han, Limit cycles near generalized homoclinic and double homoclinic loops in piecewise smooth systems, {\it Chaos, Solitons and Fractals} {\bf 45}(2012), 454--464.

\bibitem{Liang2}F. Liang, M. Han and V. G. Romanovski, Bifurcation of limit cycles by perturbing a piecewise linear Hamiltonian system with a homoclinic loop, {\it Nonlinear Anal. Ser. A} {\bf 75}(2012), 4355--4374.

\bibitem{Liu} X. Liu and M. Han, Bifurcation of limit cycles by perturbing piecewise Hamiltonian systems, {\it International Journal of Bifurcation and Chaos} {\bf 20}(2010), 1379--1390.

\bibitem{Luck} J. M. Luck and A. Mehta, Bouncing ball with a finite restitution: chattering, locking, and chaos, {\it Physical Review E} {\bf 47}(1993),  3988.

\bibitem{Meln} V. K. Melnikov,
On the stability of the center for time periodic perturbations,
{\it Trans. Moscow Math Soc.} {\bf 12}(1963), 1--57.

\bibitem{Nordmark1} A. B. Nordmark, Non-periodic motion caused by grazing incidence in an impact oscillator, {\it J. Sound Vibration} {\bf 145}(1991), 279--297.

\bibitem{NusseYorke} H. E. Nusse and J. A. Yorke, Border-collision bifurcations
including ``period two to period three" for piecewise smooth systems, {\it Physica D}
{\bf 57}(1992), 39--57.

\bibitem{Shaw1} S. W. Shaw, A. G. Haddow and S.-R. Hsieh,
 Properties of cross-well chaos in an impacting system,
{\it Phil. Trans. R. Soc. Lond. A } {\bf 347} (1994), 391--410.

\bibitem{Shaw2} S. W. Shaw and P. J. Holmes, A periodically forced piecewise linear osillator,
{\it J. Sound Vib.} {\bf 90} (1983), 129--155.

\bibitem{Shaw3} S. W. Shaw and R. H. Rand, The transition to chaos in a
simple mechanical system, {\it Int. J. Non-linear Mechanics} {\bf 24}(1989), 41--56.

\bibitem{ShenDu} J. Shen and Z. Du, Double impact periodic orbits for an inverted pendulum,
{\it Int. J. Non-linear Mechanics}  {\bf 46} (2011), 1177--1190.

\bibitem{Tufillaro} N. B. Tufillaro and A. M. Albano, Chaotic dynamics of a bouncing ball, {\it American Journal of Physics} {\bf 54}(1986), 939--944.

\bibitem{Cristina} M. C. Vargas, D. A. Huerta and V. Sosa. Chaos control: the problem of a bouncing ball revisited, {\it American Journal of Physics} {\bf 77}(2009), 857--861.

\bibitem{Vogel} S. Vogel and S. J. Linz, Regular and chaotic dynamics in bouncing ball models, {\it International Journal of Bifurcation and Chaos} {\bf 21}( 2011), 869--884.

\bibitem{Wiggins2} S. Wiggins, Chaos in the quasiperiodically forced Duffing oscillator,
{\it Physics Letters A} {\bf 124}(1987), 138--142.

\bibitem{Wiggins3} S. Wiggins, {\it Global Bifurcations and Chaos - Analytical Methods}, Springer-Verlag, New York, 1988.

\bibitem{Yagasaki} K. Yagasaki, Bifurcations and chaos in a quasi-periodically forced beam: theory, simulation and experiment,
{\it J. Sound Vibration} {\bf 183}(1995), 1--31.


\bibitem{Zou} Y. Zou and T. Kš¹pper, Generalized Hopf bifurcation emanated from a corner for piecewise smooth planar systems, {\it Nonlinear Anal. Ser. A} {\bf 62}(2005), 1--17.

\end{thebibliography}
}

\newpage
\thispagestyle{empty}
\chapter*{\heiti ¹¥¶Á˶ʿѧλÆÚŒäÍê³ÉµÄÂÛÎÄ}
\addcontentsline{toc}{chapter}{\numberline{}\mbox{\heiti
¹¥¶Á˶ʿѧλÆÚŒä¿ÆÑгɹûŒòœé}}

\begin{enumerate}
  \item ÓôŒÑ, ÎÞ×èÄᵯÌøÇòÄ£ÐÍ·¢ÉúÍêÕûÕð²üµÄÌõŒþ, ËÄŽšŽóѧѧ±š(×ÔÈ»¿Æѧ°æ)(ÒÑœÓÊÜ, Žý·¢±í).
\end{enumerate}



\newpage
\begin{appendix}
\thispagestyle{empty}

\begin{center}{\LARGE \heiti Éù\ Ã÷}\end{center}
\addcontentsline{toc}{chapter}{\numberline{}\mbox{\heiti
ÉùÃ÷}}

\vspace{0.8cm} {\large

±ŸÈËÉùÃ÷Ëù³Êœ»µÄѧλÂÛÎÄÊDZŸÈËÔÚµŒÊŠÖžµŒÏÂœøÐеÄÑП¿¹€×÷Œ°È¡µÃµÄÑП¿³É¹û¡£ŸÝÎÒËùÖª,
³ýÁËÎÄÖÐÌرðŒÓÒÔ±ê×¢ºÍÖÂлµÄµØ·œÍâ,
ÂÛÎÄÖв»°üº¬ÆäËûÈËÒÑŸ­·¢±í»ò׫Ў¹ýµÄÑП¿³É¹û, Ò²²»°üº¬Îª»ñµÃ
ËÄŽšŽóѧ»òÆäËûœÌÓý»ú¹¹µÄѧλ»òÖ€Êé¶øʹÓùýµÄ²ÄÁÏ¡£ÓëÎÒһͬ¹€×÷µÄͬ֟¶Ô±ŸÑП¿Ëù×öµÄÈκι±Ï×
ŸùÒÑÔÚÂÛÎÄÖÐ×÷ÁËÃ÷È·µÄ˵Ã÷²¢±íʟлÒâ¡£

±ŸÑ§Î»ÂÛÎijɹûÊDZŸÈËÔÚËÄŽšŽóѧ¶ÁÊéÆÚŒäÔÚµŒÊŠÖžµŒÏÂÈ¡µÃµÄ,
ÂÛÎijɹû¹éËÄŽšŽóѧËùÓÐ, ÌØŽËÉùÃ÷¡£

\vspace{3cm} \hspace*{7.5cm} µŒÊŠ\underline{\hspace{3.5cm}}

\vspace{0.7cm} \hspace*{7.5cm} ×÷Õß\underline{\hspace{3.5cm}}

\vspace{0.6cm} \hspace*{7.5cm} ¶þÁãÒ»ÎåÄêÈýÔÂÊ®¶þÈÕ}

\end{appendix}

\end{document}
